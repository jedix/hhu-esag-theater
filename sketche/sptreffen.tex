%SP-Treffen

\newpage
\section{SP-Treffen}
\label{sec:sp-treffen}
    \charaktere{\Frodo, \Sum, \Gimli, \Legolars, \Elron, \Gandalf, \Galadriel}
    \setting{Riesentischkreis}
    \hauptbeamer{nichts}
    \sound{nichts}
    \licht{normal oder Spot auf Tisch}
    \requisiten{Tisch, Stühle, PO-Kette, Laptops, Schrottwichtelgeschenke}
    
\regie{Das Studierendenparlament tagt. TOP-Liste:\\
TOP 0: Regularia
TOP 1: AstA-Berichte
TOP n: Schrottwichteln
TOP n+1: Sonstiges
}

\begin{verseplay}[10em]
\s{\Elron} Herzlich Willkommen zur 34. Sondersitzung des Studierendenparlamentes, stattfindend zwischen der regulären 1465. und 1466. Sitzung, auf Antrag von Frodo Beutlin, Student der mathematisch-naturwissenschaftlichen Fakultät, fernmündlich eingereicht über die Facebook-Pinnwand des Studierendenparlamentes. Ich schlage \Galadriel als Protokollführende vor. Gibt es Gegenstimmen? \kregie{Pause} Gut. Kommen wir zur Feststellung der Anwesenheit. \Galadriel ist anwesend. \Sum ?

\regie{Galadriel tippt die ganze Zeit über Protokoll.}

\s{\Sum} Ja.
\s{\Elron} G.. 1.. mli?
\s{\Gimli} Das wird Gimli ausgesprochen, jo.
\s{\Elron} \Legolars ?
\s{\Legolars} Anwesend. \kregie{meldet sich mit Bogen}
\s{\Elron} \Gandalf ?
\s{\Frodo} Der kommt nach.
\s{\Elron} Wie immer. \Frodo ?
\s{\Frodo} Nö.
\s{\Elron} Echt jetzt? \Saruman ? \kregie{Pause} Der AstA-Vorstand ist wie üblich nicht anwesend. Ich stelle fest, dass wir trotzdem beschlussfähig sind. Gibt es Wünsche zur Änderung der bekanntgegebenen TOP-Liste? \kregie{Pause, Frodo guckt ausdruckslos in der Gegend herum} \Frodo?
\s{\Frodo} Äh, ja. \Gandalf hat gesagt, ich solle schonmal eine Sondersitzung einberufen, es geht um irgendeine neue Rahmenprüfungsordnung.
\s{\Elron} Eine neue Prüfungsordnung wäre doch in den amtlichen Bekanntmachungen aufgeführt worden?
\regie{Frodo zuckt mit den Schultern.}
\s{\Elron} Nun gut, da Gandalf noch nicht erschienen ist, füge ich den TOP Rahmenprüfungsordnung ans Ende der TOP-Liste vor Sonstiges an. Weitere Änderungswünsche? \kregie{Pause} Gut. TOP 1, Berichte des AstA. \kregie{Pause} Offensichtlich hat der AstA heute nichts zu berichten.

\regie{Gandalf stürmt auf Steckenpferd herein.}

\s{\Gandalf} Wir sind alle dem Untergang geweiht! Der Rektor hat \kregie{wird HIER unterbrochen} eine neue Rahmenprüfungsordnung...
\s{\Elron} Einen Moment. Ich stelle hiermit für das Protokoll Gandalfs verspätete Anwesenheit fest. Wärest du pünklich erschienen, so wäre dir bekannt, dass der TOP Rahmenprüfungsordnung an das Ende der TOP-Liste gesetzt wurde.
\s{\Gandalf} Aber aber aber...
\s{\Elron} \kregie{Unbeeindruckt} Wir fahren nun fort mit TOP n: Schrottwichteln.
\regie{Gandalf setzt sich grummelnd hin.}
\s{\Frodo} Ich habe doch garnichts fürs Schrottwichteln dabei.
\s{\Elron} Das Schrottwichteln ist eine lange ungebrochene Tradition des Studierendenparlamentes. \kregie{Funkelt \Frodo böse an.}
\s{\Sum} Guck einfach, ob du irgendwas in deinen Taschen dabei hast.
\regie{\Frodo kramt. Jeder liegt verdeckt etwas in eine Kiste, die auf den Tisch gestellt wird. In der Kiste: Goldkette, blaues Schwert, Limburger-und-Sardellen-Sandwich (von \Frodo), 1-UP-Pilz, goldener Ring mit komischer elbischer Inschrift und Benzin für die Kettensäge.}

\regie{\Gandalf steht auf, zieht 1-UP-Pilz aus der Kiste}
\s{\Gandalf} Uhhhh, ein Pilz. \kregie{setzt sich wieder}

\regie{\Sum steht auf, zieht blaues Schwert aus der Kiste}
\s{\Gimli} Was ist das?
\s{\Sum} Ein blaues Schwert.
\s{\Gimli} Und was tut es?
\s{\Sum} Es leuchtet blau. \kregie{setzt sich wieder hin}

\regie{\Frodo steht auf, zieht Goldkette aus der Kiste}
\s{\Frodo} Oh, eine Goldkette. \kregie{setzt sich wieder hin}

\regie{\Gimli steht auf, zieht Limburger-und-Sardellen-Sandwich}
\s{\Gimli} \kregie{empört} Was ist das denn?
\s{\Frodo} Ehmmm... das erkenne ich sofort: Ein Limburger-und-Sardellen-Sandwich. Ein Bissen, und du musst 4 Tage lang nichts essen.
\s{\Elron} \kregie{wirft ein} \emph{Möchtest} 4 Tage lang nichts essen.
\regie{\Gimli setzt sich wieder}

\regie{\Legolars steht auf, zieht Benzin}
\s{\Legolars} \kregie{Liest vor} Benzin für die Kettensäge... \kregie{enttäuscht} Toll, Nahkampfwaffenmunition. \kregie{Setzt sich wieder hin.}

\regie{\Elron steht auf, zieht Ring}
\s{\Elron} Oh, ein goldener Ring mit elbischer Inschrift.
\s{\Sum} Was steht drauf? Was steht drauf?
\s{\Elron} Es ist in der dunklen Sprache geschrieben. \kregie{kurze Pause} Made in Bangladesh.
\regie{\Elron steckt den Ring in die Tasche, räumt die Kiste vom Tisch, setzt sich wieder.}

\s{\Elron} Nun, nachdem die Tradition des Schrottwichtelns bei Sitzungen, auch außerordentlichen, des Studierendenparlamentes gewahrt wurde, kommen wir zum nächsten TOP: Rahmenprüfungsordnung und Zerstörung der Welt.
\s{\Gandalf} Wie Frodo euch sicher bereits mitgeteilt hat, hat der Rektor ohne das Wissen der Studenten eine weitere Prüfungsordnung erschaffen.
\regie{Frodo zeigt Prüfungsordnung}
\s{\Gandalf} Diese wird, wenn sie in Kraft tritt, über allen anderen Prüfungsordnungen stehen und mit ihren Regeln alles Leben unterwerfen. Das Studentenleben, wie wir es kennen, wird dann nicht mehr möglich sein.
\s{\Elron} Nein.
\s{\Gandalf} Doch.
\s{Chor ausser \Gandalf} Ohhhh!
\s{\Elron} \kregie{arrogant} Ohne die Unterschrift des AstA-Vorsitzenden wird diese Prüfungsordnung nicht in Kraft treten können.
\s{\Gandalf} Der AstA-Vorsitzende ist uns keine Hilfe. Er hat bereits unterzeichnet.
\s{\Elron} Nein.
\s{\Gandalf} Doch.
\s{Chor ausser \Gandalf} Ohhhh!
\s{\Frodo} Kann ich sie nicht einfach kaputt machen? In den Müll werfen?
\s{\Gandalf} Nein. Sie kann nur da zerstört werden, wo sie geschaffen wurde. Wir müssen ins Rektorat.
\s{\Elron} Man geht nicht einfach ins Rektorat.
\s{\Gimli} Dann brauchen wir einen Plan. Und eine Gruppe wackerer Helden.
\s{\Elron} Ich werde von hier aus mit \Galadriel den Widerstand organisieren.

\s{\Frodo} Ich habe die Prüfungsordnung, ich werde sie auch vernichten. \kregie{Bindet Prüfungsordnung an Kette}
\s{\Sum} YOLO! Ich bin dabei.
\s{\Gandalf} Auf mich werdet ihr nicht verzichten können, ich kenne alle Wege zum Rektorat.
\s{\Legolars} Du kannst mit meinem Bogen rechnen. \kregie{Hält Bogen hoch}
\s{\Gimli} Und mit meiner Axt! Und meinem Sandwich! \kregie{Hält erst Axt, dann auch Sandwich hoch}

\regie{Licht aus.}

\end{verseplay}
