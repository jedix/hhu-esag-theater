\newpage
\section{Vorlesung}
\label{sec:vorlesung}
\charaktere{\Sum, \Gimli, \Prof, \Studa, \Studb, \Studc}
\setting{Leere Tafel. Star Gate Möglichst neben der ``Professorentafel'' Die Stühle stehen auf der anderen Seite als Seminaraumsetting.}
\hauptbeamer{-}
\sound{-}
\licht{normal}
\requisiten{Star Gate, Fünf Stühle, Pult, Kreide}
\regie{\Prof steht an der Tafel \Studa, \Studb, \Studc Sitzen auf den Stühlen. \Gimli und \Sum sind hinter dem Stargate versteckt.}    

\begin{verseplay}[7em]
\s{\Prof} Hallo liebe Studenten im heutigen Seminar zu den komperativen Vergleichswissenschaften. Wie ich in der letzten Vorlesung schon angedeutet hatte, möchten wir uns heute mit Überbegrifflichkeiten beschäftigen.
\end{verseplay}
\regie{geht zur Tafel und zeichnet an "Ü"}

\begin{verseplay}[7em]
 \s{\Prof} Wenn man nicht ins Detail gehen kann, sind Überbegrifflichkeiten notwendig. So ist z.b. das "Dingsgedingsel" eine Überbegrifflichkeit, die alles bezeichnen kann, was bei Fünf nicht auf den Bäumen ist. Aber hier sehen wir auch schon die ersten Grenzen. Denn, möchte man in der Zoologie der Eichhörnchen und anderer Baumbewohner ungenau sein, ist dieses Wort schlecht einzusetzen. Man kann "Dingsgedingsel" in den vergleichenden Baumsäugerwissenschaften geradezu als Synonym für Faultiere nutzen.
\end{verseplay}
\regie{\Sum und \Gimli kommen durch das Star Gate}
\begin{verseplay}[7em]
\s\Prof Wenn sie schon zu Spät kommen, können sie wenigstens so freundlich sein, das Star Gate im Hinteren Raum zu Nutzen. Setzen sie sich jetzt bitte hin.
\s\Gimli \kregie{erschrocken} Wo zum Teufel sind wir hier?
\s\Prof Fragen beantworte ich gerne nach dem Seminar.
\s\Gimli Aber wir wollten doch nur \kregie{wird HIER unterbrochen} ins Rekorat.
\s\Prof \kregie{nachdrücklich} Setzen sie sich bitte.
\end{verseplay}
\regie{\Sum und \Gimli lassen sich einschüchtern und setzen sich auf die freien Plätze}

\begin{verseplay}[7em]
\s\Prof Wo war ich stehen geblieben?
\s\Studa Beim Gedingseldings.
\s\Studb Er meint Dingsgedingsel, glaube ich.
\s\Prof Ah, sehr gut. Wie sie an diesem Beispiel sehen, gibt es bei Umschreibungsworten einige Fallen. Nehmen wir zum Beispiel das Dingsbumms.
\end{verseplay}
\regie{Geht zur Tafel und schreibt Dingsbumms an.}


\begin{verseplay}[7em]
\s\Prof \kregie{während er schreibt} Das Dingsbumms ist eine übergeordnete Sprachkonstruktion, die so ziemlich alles bedeuten kann, währenddessen das Bummsdings \kregie{schreibt Bummsdings daneben} in seiner Beschreibungsvariabilität deutlich eingeschränkter ist. Kann mir jemand ein Beispiel für ein Dingsbumms nennen?

\s\Studc Eine Melone! \kregie{\Prof schreibt das unter Dingsbumms}
\s\Studa Eine kalte Ente! \kregie{\Prof schreibt das unter Dingsbumms}
\s\Studb Ein Motorrad! \kregie{\Prof schreibt das unter Dingsbumms}
\s\Prof Genau. Wobei Das Motorrad auch unter Bummsdings fallen kann, wenn die Zündkerze nicht richtig eingestellt ist. Können sie sich auch etwas unter Bummsdings vorstellen?
\s\Studa Knaller! \kregie{\Prof schreibt das unter Bummsdings}
\s\Studb Prellbock! \kregie{\Prof schreibt das unter Bummsdings}
\s\Gimli Dynamit! \kregie{\Prof schreibt das unter Bummsdings}
\s\Prof Sehr Gut. Dingsbummse können also jedwede Art von Ding sein, während ein Bummsdings ein Ding sein muss das entweder bummst, das gebummst wird oder das bummsen oder gebummst werden kann. 
\s\Studb Herr Professor?
\s\Prof Ja bitte.
\s\Studb Sehe ich das richtig, das eine Knallerbse ein Bummsdings ist? Schliesslich ist der Knaller ja auch ein Bummsdings.
\s\Prof \kregie{überlegt kurz} Die Knallerbse ist tatsächlich kein Bummsdings. Das Knallen des Knallers lässt sich zwar als ``Bumms'' umschreiben, das Knallen einer Knallerbse geht jedoch eher in Richtung Knall.
\s\Studb Dann ist also Poppkorn kein Bumsdings weil es ja gepoppt aber nicht gebummst wird.
\s\Prof Richtig. Auch wenn Poppen und Bumsen zur gleichen Geräuschfamilie gehören, wie übrigens auch das Knattern, muss beim Bummsdings hier genau unterschieden werden. Hat noch jemand Fragen?
\s\Studa Im letzten Übungsblatt, da wo wir Äpfel mit Birnen vergleichen sollten kommt bei mir immer Ernte raus.
\s\Prof Oh, da sollte eigentlich Ente herauskommen \kregie{geht hin und fängt an zu lesen} Das nehm ich eben mit ins Büro und geb ihnen danach Bescheid, vielleicht hat mein Assistent einen Fehler gemacht.
\s\Studa \kregie{zu \Gimli und \Sum} Kommt ihr noch mit auf ein Bier?
\s\Gimli Aber wir müssen dringend zum \kregie{wird unterbrochen} Rektorat.
\s\Sum Och komm, ein Bier geht doch...
\s\Studb Super, kommt.

\end{verseplay}

\regie{Licht aus.}
