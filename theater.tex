\documentclass{scrartcl}
\usepackage[utf8]{inputenc}
\usepackage[T1]{fontenc}
\usepackage[ngerman]{babel}
\usepackage{play}
\usepackage{titlesec}
\usepackage{hyperref}
\usepackage{amsmath}
\usepackage{xspace}
\titleformat{\section}{\bf}{Szene \thesection:\quad}{0em}{}
\itemindent 0pt
%
%        BEGIN: Utilities
%
\newcommand{\s}[1]{\speaker #1}
\newcommand{\regie}{\longdirection}
\newcommand{\kregie}{\shortdirection}
\newcommand{\charaktere}[1]{\regie{\textbf{Charaktere:} #1}}
\newcommand{\setting}[1]{\regie{\textbf{B{\"u}hnenbild:} #1}}
\newcommand{\hauptbeamer}[1]{\regie{\textbf{Hauptbeamer:} #1}}
\newcommand{\sound}[1]{\regie{\textbf{Sound:} #1}}
\newcommand{\licht}[1]{\regie{\textbf{Licht:} #1}}
\newcommand{\requisiten}[1]{\regie{\textbf{Requisiten:} #1}}
\newcommand{\technik}{\xspace\shortdirection}
%g\newcommand{\technik}[1]{}
%
%        END: Utilities
%


%
%        BEGIN:  Charaktere
%

% Hauptcharaktere
\newcommand{\Frodo}{Frodo\xspace}
\newcommand{\Gandalf}{Gandalf\xspace}
\newcommand{\Sum}{Sum\xspace}
\newcommand{\Legolars}{Legolars\xspace}
\newcommand{\Gimli}{G1ml1\xspace}
\newcommand{\Monk}{Monk\xspace}
\newcommand{\Elron}{Elron\xspace}
\newcommand{\Galadriel}{Galadriel\xspace}
%Nebencharaktere
\newcommand{\Saruman}{Saruman\xspace}
\newcommand{\Pacman}{Pacman\xspace}
\newcommand{\Waldo}{Waldo\xspace}
\newcommand{\Baum}{Baum\xspace}
\newcommand{\Firmenname}{Richtig tolle Firma GmbH und Co KG}
\newcommand{\Beamtenzombies}{Beamtenzombies\xspace}

\newcommand{\Tuermonteura}{Türmonteur 1\xspace}
\newcommand{\Tuermonteurb}{Türmonteur 2\xspace}
\newcommand{\Leichen}{Leichen\xspace}
\newcommand{\Petrus}{Petrus\xspace}
\newcommand{\Euler}{Euler\xspace}
\newcommand{\Waschmaschine}{Waschmaschine\xspace}
\newcommand{\Chor}{Chor\xspace}
\newcommand{\Sing}{Sing\xspace}
\newcommand{\Alle}{Alle\xspace}

% QED
\newcommand{\QEDHost}{George \xspace}
\newcommand{\QEDGuestA}{QEDGuestA\xspace}
\newcommand{\QEDGuestB}{QEDGuestB\xspace}
\newcommand{\QEDGuestC}{QEDGuestC\xspace}

\newcommand{\Prof}{Professor\xspace}
\newcommand{\Studa}{Student 1\xspace}
\newcommand{\Studb}{Student 2\xspace}
\newcommand{\Studc}{Student 3\xspace}

\newcommand{\Paul}{Paul}
\newcommand{\Spock}{Spock}
\newcommand{\Pille}{Pille}

%
%        END:  Charaktere
%

\begin{document}
%
%        BEGIN: Titelseite
%
\title{ESAG-Theater 2012}
\author{}
\date{Letzte "Anderung: \today}
\maketitle
\tableofcontents
%
%        END:        Titelseite
%
%
%        BEGIN: Zusammenfassung Charaktere
%
\newpage
\textbf{Charaktere}
%\begin{verseplay}[10em]
%\s{\Frodo}
%        \kregie{Schauspieler} \kregie{Szene \ref{sec:frodo-findet-po}}\\
%        Kurzbeschreibung
%\end{verseplay}

\newpage
\section{Band: Intro}
\label{sec:band_intro}
\charaktere{Band}
\setting{Band}
\hauptbeamer{-}
\sound{-}
\licht{Band}
\requisiten{-}

\regie{Die Band spielt solange Quark und andere Milchprodukte, bis das Theater anfangen kann.}

\regie{Licht aus.}

\newpage
\section{Kazoo-Intro}
\label{sec:kazoo_intro}
\charaktere{Kazooisten}
\setting{-}
\hauptbeamer{THX-Logo, Dolby Analog}
\sound{-}
\licht{-}
\requisiten{-}
    
\regie{Die Kazooisten sind im Raum verteilt. Wenn das THX-Logo erscheint, spielen sie den THX-Sound. Danach spielen die das Harry-Potter-Intro.}

\newpage
\section{Vorspann}
\label{sec:vorspann}
\charaktere{-}
\setting{-}
\hauptbeamer{Video: Intro}
\sound{-}
\licht{-}
\requisiten{-}

\regie{Video: Star-Wars-Schriftzug}
\begin{verseplay}[7em]
\s{{Schriftzug}}
Es war einmal vor langer Zeit in einer weit, weit entfernten Galaxis...\\
Das Universum ist im Wandel.\\
Ich spüre es im Vakuum, \\
ich spüre es im Raum,\\
ich rieche es im Äther.\\
Vieles was einst war ist verloren,\\
da niemand mehr lebt, der sich erinnert.\\
Alles begann mit dem Schmieden der großen Prüfungsordnungen...
\end{verseplay}

\regie{Video: Original Herr-der-Ringe-Intro\\
Off: Düstere geheimnisvolle Frauen-Stimme aus dem Off\\
Chor: Düsterer geheimnisvoller Chor aus dem Off\\
Off 2: Überschwängliche Stimme aus dem Off}

\begin{verseplay}[4em]
\s{{Off}}
Drei wurden der Inphima gegeben, unsterblich, und die weisesten und reinsten aller Lebewesen.\\
Sieben den Medizinern, große Quacksalber und geschickte Metzger, in ihren Hallen aus Backstein.\\
Und neun, neun Prüfungsordnungen wurden den Wirtschaftswissenschaftlern geschenkt, die vor allem anderen nach Macht streben.\\
Denn diese Prüfungsordnungen bargen die Stärke und den Willen jede Fakultät zu leiten.\\
Doch sie wurden alle betrogen.\\
Denn es wurde noch eine Prüfungsordnung gefertigt...\\
In der Verwaltung, in der Tiefe des Rektorats, schmiedete der dunkle Rektor Sauron heimlich eine Rahmenprüfungsordnung, um alle anderen zu beherrschen.\\
In diese Prüfungsordnung floss seine Grausamkeit, seine Bosheit, und sein Wille alles Leben zu unterdrücken.\\
\s{{Chor}}
Eine Prüfungsordnung sie zu knechten,\\
sie alle zu finden,\\
ins Dunkel zu treiben,\\
und ewig zu binden.\\
\s{{Off}}
Min Gôr bauglo hain phain,\\
min Gôr chebo hain,\\
min Gôr togo hain phain,\\
ned duir gwedho hain.\\
\s{{Off 2}}Das war Elbisch.
\end{verseplay}

\regie{Video: Melodie der Sendung mit der Maus.}

%Frodo Findet Prüfungsordnung

\newpage
\section{Frodo Findet Pruefungsordnung}
\label{sec:frodo-findet-po}
    \charaktere{\Frodo, \Gandalf }
     \setting{Poststelle}
    \hauptbeamer{nichts}
    \sound{}
    \licht{}
    \requisiten{Kaffee (Tasse), Briefkasten, Post, Prüfungsordnung}
    
\regie{Frodo holt seine Hauspost ab und hält dabei einen Kaffe in der Hand. Frodo trägt einen Bademantel. Er findet in seinem Postfach die Rahmenprüfungsordnung.}
\begin{verseplay}[4em]
\s{\Frodo} Rechnung, Mahnung, Spam, Rechnung, Spam, Mahnung, Spam, Spam, Spam, Bacon, Eggs and Spam. 
			\kregie{Wirft Rechnungen weg und behält Spam. Findet Paketabhohlschein mit beschriftung "Wichtig".}
			Warum kann die Hauspost nicht zu normalen Zeiten liefern. Jetzt muss ich wieder zu Gandalf latschen und mein Paket holen.
			\kregie{Stampft wütend zu Gandalfs Tür und klopft an.}
\s{\Gandalf} \kregie{Gandalf öffnet die Tür. Man hört Raggae/Gabba-musik. Er nimmt den Paketschein.}
				Wat is dat denn? Ahh das Paket was diese bescheuerte Hauspost eben gebracht hat. 
\regie{ Gandalf gibt \Frodo übertrieben großes Paket. \Frodo bricht unter der Last zusammen.}
\s{\Frodo} Danke \kregie{Ironisch}
\s{\Gandalf} Nichts für.
\regie{\Frodo packt das Paket aus. Eine Menge Verpackungsmaterial kommt zum Vorschein. Am Boden des Pakets findet er einen Stapel Papier mit unbekannten Schriftzeichen. Frodo dreht und wendet das Papier, blättert es durch und schaut verdutzt. }
\s{\Frodo} Was ist das?
\regie{Gandalf reißt Frodo das Papier aus der Hand und begutachtet es. Währendessen untersucht Frodo die Verpackung.}
\s{\Frodo} Dekanat? Das ist ja gar nicht für mich.
\s{\Gandalf} \kregie{Gandalf packt die Panik} 
				Das sieht seltsam aus. Lass uns das mal genauer angucken.
\regie{Beide setzen sich an den Tisch. Gandalf zieht Bier aus dem Bart und reicht eins an Frodo. Dann zieht er sich Erdnüsse aus dem Bart.}
\s{\Gandalf} Jetzt erstmal ein paar Erdnüsse. Aber nur die Guten von \Firmenname.
\regie{ Gandalf präsentiert die Ernüsse und legt sie auf den Tisch. Frodo unterstützt ihn bei der Präsentation. Er liest einen Moment konzentriert den Papierstapel und schlägt in einem Ordner nach, der sich schon vorher auf dem Tisch befand. Frodo trinkt Bier und starrt in der Gegend rum.}
\s{\Gandalf} Das ist eine wirklich ernste Angelegenheit. 
\s{\Frodo} Was steht denn drauf?
\s{\Gandalf} \kregie{Liest die Vorderseite vor}
					Eine Prüfungsordnung sie zu knechten.
					Sie alle zu finden,
					ins dunkel zu treiben
					und ewig zu binden.
\s{\Frodo} Das klingt jetzt nicht so gut.
\s{\Gandalf} Das hast du gut erkannt.
                 \kregie{Blättert durch die Prüfungsordnung und liest vor}
				Guck mal hier, Paragraph 19, Absatz 7. Überschreiten der Regelstudienzeit um mehr als ein Semester führt zur automatischen Exmatrikulation.
				Oder hier, \kregie{zeigt auf Textstelle} ganz schlimm, Paragraph 9$\dfrac{3}{4}$, Absatz $\pi$, keine Abmeldemöglichkeit für Klausuren, auch nicht im Krankheitsfall.
\s{\Frodo} Das klingt ja wirklich nicht so gut.
\s{\Gandalf} Ich muss mal den AstA-Vorsitzenden \Saruman fragen, was es damit auf sich hat. Ruf das Studierendenparlament zusammen, am besten über Facebook, es muss schnell gehen.
\regie{Gandalf geht hastig ab. Licht aus.}
			 
\end{verseplay}

%ideen:
%Steht am Briefkasten und sortiert seine Post. Rechnungen, Mahnungen, Spam, Spam, Spam, Bacon, Eggs and Spam.
%Ein großer Umschlag, hmmm, ans Dekanat, muss wichtig sein, schulterzuck, auf.
%Gandalf kommt an. Wat is dat denn? bla bla bla Rahmenprüfungsordnung bla bla vernichtet werden.
%Evtl. Bier und Erdnüsse im Bart (Anhalteranspielung). Muss vernichtet werden bla. Da wo sie erstellt wurde, im Rektorat.
%Könnte sehr gefährlich sein, ich erkundige mich bei meinem Dekan, du ruft schon mal ne SP-Saal-Sitzung (sic!) ein.

%Treffen zwischen Gandalf und Saruman

\newpage
\section{Treffen zwischen Gandalf und Saruman}
\label{sec:gandalf-saruman}
    \charaktere{\Gandalf, \Saruman }
    \setting{Sarumans Büro}
    \hauptbeamer{nichts}
    \sound{nichts, später Lichtschwerteffekt mit anschließendem Duel of the fates}
    \licht{normal, später UV}
    \requisiten{Wanderstöcke/Lichtschwerter (UV-Farb-lackierte Besenstile)}
    
\regie{Saruman sitzt am Tisch, mehrere Papierstapel stehen vor ihm. Laptop steht vor ihm, tippt, wirkt beschäftigt. Plasmalampe steht daneben.}
\regie{Unsichtbare Tür wird auf die Bühne getragen und montiert.}

\begin{verseplay}[10em]
\regie{Gandalf betritt die Bühne, klopft an unsichtbare Tür}
\s{\Saruman} Wer ist da?
\s{\Gandalf} Feuer
\s{\Saruman} Feuer-wer?
\s{\Gandalf} \kregie{Steckt Kopf durch die Tür} Wo brennts denn?
\s{\Saruman} Aaaaarrrrrgggghhhhh... \kregie{Facepalm}
\regie{Gandalf tritt ein.}
\s{\Saruman} Je-des-mal. Was kann ich für dich tun, Ghandi?
\s{\Gandalf} Es gibt noch eine weitere Prüfungsordnung, der Rektor hat uns alle betrogen. Sie dient der Unterdrückung allen Lebens, alle anderen Prüfungsordnungen sind ihr unterstellt.
\s{\Saruman} Ich weiß.
\s{\Gandalf} \kregie{Verwundert} Du weißt?
\s{\Saruman} Du siehst das alles falsch, sie kann uns nutzen. Wir können sie für unsere Zwecke einsetzen und alle Wirtschaftswissenschaftler, Juristen und Mediziner verbannen.
\s{\Gandalf} Der Rektor wird niemals etwas von seiner Macht an die Studierenden abtreten, auch nicht an den AstA-Vorsitzenden.
\s{\Saruman} Es ist zu spät, ich habe sie schon in die Hauspost gegeben.
\s{\Gandalf} Ach so ist das... ehh... Ich geh dann mal.
\s{\Saruman} Haaaaaaalt Stop! Schließe dich der dunklen Seite der Macht an, zusammen können wir die Universität beherrschen.
\s{\Gandalf} Ich werde mich dir niemals anschließen.
\s{\Saruman} Dann musst du sterben.

\regie{Lichtschwerter werden gezückt. UV-Licht an. Soundeffekt. Musik: Duel of the fates. Epischer Lichtschwertkampf. Und mit episch meine ich episch. Echt jetzt. Mit schwarz gekleideten Leuten, die bei Salti helfen und so. Saruman stirbt. Gandalf geht ab.}

\end{verseplay}

\newpage
\section{Der Physiker}
\label{sec:physiker}
\charaktere{Physichor}
\setting{-}
\hauptbeamer{-}
\sound{-}
\licht{normal}
\requisiten{-}
    
\regie{Same procedure as every year.}

\regie{Licht aus.}

\newpage
\section{SP-Treffen}
\label{sec:sp-treffen}
\charaktere{\Frodo, \Sum, \Gimli, \Legolars, \Elron, \Gandalf, \Galadriel}
\setting{Riesentischkreis}
\hauptbeamer{-}
\sound{-}
\licht{normal oder Spot auf Tisch}
\requisiten{Tisch, Stühle, PO-Kette, Laptops, Schrottwichtelgeschenke}
    
\regie{Das Studierendenparlament tagt. TOP-Liste:\\
TOP 0: Regularia\\
TOP 1: AstA-Berichte\\
TOP n: Schrottwichteln\\
TOP n+1: Sonstiges
}

\begin{verseplay}[7em]
\s{\Elron} Herzlich Willkommen zur 34. Sondersitzung des Studierendenparlamentes, stattfindend zwischen der regulären 1465. und 1466. Sitzung, auf Antrag von \Frodo Beutlin, Student der mathematisch-naturwissenschaftlichen Fakultät, fernmündlich eingereicht über die Facebook-Pinnwand des Studierendenparlamentes. Ich schlage \Galadriel als Protokollführende vor. Gibt es Gegenstimmen? \kregie{Pause} Gut. Kommen wir zur Feststellung der Anwesenheit. \Galadriel ist anwesend. \Sum?
\end{verseplay}
\regie{\Galadriel tippt die ganze Zeit über Protokoll.}
\begin{verseplay}[7em]
\s{\Sum} Ja.
\s{\Elron} G.. 1.. m.. l.. 1..?
\s{\Gimli} Das wird Gimli ausgesprochen, jo.
\s{\Elron} \Legolars?
\s{\Legolars} Anwesend. \kregie{meldet sich mit Bogen}
\s{\Elron} \Gandalf?
\s{\Frodo} Der kommt nach.
\s{\Elron} Wie immer. \Frodo?
\s{\Frodo} Nö.
\s{\Elron} Echt jetzt? \Saruman? \kregie{Pause} Der AstA-Vorstand ist wie üblich nicht anwesend. Ich stelle fest, dass wir trotzdem beschlussfähig sind. Gibt es Wünsche zur Änderung der bekanntgegebenen TOP-Liste? \kregie{Pause, \Frodo guckt ausdruckslos in der Gegend herum} \Frodo?
\s{\Frodo} Äh, ja. \Gandalf hat gesagt, ich solle schonmal eine Sondersitzung einberufen, es geht um irgendeine neue Rahmenprüfungsordnung.
\s{\Elron} Eine neue Prüfungsordnung wäre doch in den amtlichen Bekanntmachungen aufgeführt worden?
\s{\Frodo} \kregie{zuckt mit den Schultern}
\s{\Elron} Nun gut, da Gandalf noch nicht erschienen ist, füge ich den TOP Rahmenprüfungsordnung ans Ende der TOP-Liste vor Sonstiges an. Weitere Änderungswünsche? \kregie{Pause} Gut. TOP 1, Berichte des AstA. \kregie{Pause} Offensichtlich hat der AstA heute nichts zu berichten.
\end{verseplay}
\regie{Gandalf stürmt auf Steckenpferd herein.}
\begin{verseplay}[7em]
\s{\Gandalf} Wir sind alle dem Untergang geweiht! Der Rektor hat \kregie{wird HIER unterbrochen} eine neue Rahmenprüfungsordnung...
\s{\Elron} Einen Moment. Ich stelle hiermit für das Protokoll \Gandalf{}s verspätete Anwesenheit fest. Wärest du pünklich erschienen, so wäre dir bekannt, dass der TOP Rahmenprüfungsordnung an das Ende der TOP-Liste gesetzt wurde.
\s{\Gandalf} Aber aber aber...
\s{\Elron} \kregie{unbeeindruckt} Wir fahren nun fort mit TOP n: Schrottwichteln.
\end{verseplay}
\regie{\Gandalf setzt sich grummelnd hin.}
\begin{verseplay}[7em]
\s{\Frodo} Ich habe doch garnichts fürs Schrottwichteln dabei.
\s{\Elron} Das Schrottwichteln ist eine lange ungebrochene Tradition des Studierendenparlamentes. \kregie{funkelt \Frodo böse an}
\s{\Sum} Guck einfach, ob du irgendwas in deinen Taschen dabei hast.
\end{verseplay}
\regie{\Frodo kramt. Jeder liegt verdeckt etwas in eine Kiste, die auf den Tisch gestellt wird. In der Kiste: Goldkette, blaues Schwert, Limburger-und-Sardellen-Sandwich (von \Frodo), 1-UP-Pilz, goldener Ring mit komischer elbischer Inschrift und Benzin für die Kettensäge.}
\regie{\Gandalf steht auf, zieht 1-UP-Pilz aus der Kiste.}
\begin{verseplay}[7em]
\s{\Gandalf} Uhhhh, ein Pilz. \kregie{setzt sich wieder}
\end{verseplay}
\regie{\Sum steht auf, zieht blaues Schwert aus der Kiste.}
\begin{verseplay}[7em]
\s{\Gimli} Was ist das?
\s{\Sum} Ein blaues Schwert.
\s{\Gimli} Und was tut es?
\s{\Sum} Es leuchtet blau. \kregie{setzt sich wieder hin}
\end{verseplay}
\regie{\Frodo steht auf, zieht Goldkette aus der Kiste.}
\begin{verseplay}[7em]
\s{\Frodo} Oh, eine Goldkette. \kregie{setzt sich wieder hin}
\end{verseplay}
\regie{\Gimli steht auf, zieht Limburger-und-Sardellen-Sandwich.}
\begin{verseplay}[7em]
\s{\Gimli} \kregie{empört} Was ist das denn?
\s{\Frodo} Ehmmm... das erkenne ich sofort: Ein Limburger-und-Sardellen-Sandwich. Ein Bissen, und du musst 4 Tage lang nichts essen.
\s{\Elron} \kregie{wirft ein} \emph{Möchtest} 4 Tage lang nichts essen.
\end{verseplay}
\regie{\Gimli setzt sich wieder.}
\regie{\Legolars steht auf, zieht Benzin.}
\begin{verseplay}[7em]
\s{\Legolars} \kregie{Liest vor} Benzin für die Kettensäge... \kregie{enttäuscht} Toll, Nahkampfwaffenmunition. \kregie{setzt sich wieder hin}
\end{verseplay}
\regie{\Elron steht auf, zieht Ring.}
\begin{verseplay}[7em]
\s{\Elron} Oh, ein goldener Ring mit elbischer Inschrift.
\s{\Sum} Was steht drauf? Was steht drauf?
\s{\Elron} Es ist in der dunklen Sprache geschrieben. \kregie{kurze Pause} Made in Bangladesh.
\end{verseplay}
\regie{\Elron steckt den Ring in die Tasche, räumt die Kiste vom Tisch, setzt sich wieder.}
\begin{verseplay}[7em]
\s{\Elron} Nun, nachdem die Tradition des Schrottwichtelns bei Sitzungen, auch außerordentlichen, des Studierendenparlamentes gewahrt wurde, kommen wir zum nächsten TOP: Rahmenprüfungsordnung und Zerstörung der Welt.
\s{\Gandalf} Wie \Frodo euch sicher bereits mitgeteilt hat, hat der Rektor ohne das Wissen der Studenten eine weitere Prüfungsordnung erschaffen.
\s{\Frodo} zeigt die Prüfungsordnung
\s{\Gandalf} Diese wird, wenn sie in Kraft tritt, über allen anderen Prüfungsordnungen stehen und mit ihren Regeln alles Leben unterwerfen. Das Studentenleben, wie wir es kennen, wird dann nicht mehr möglich sein.
\s{\Elron} Nein.
\s{\Gandalf} Doch.
\s{{Chor}} Ohhhh!
\s{\Elron} \kregie{arrogant} Ohne die Unterschrift des AstA-Vorsitzenden wird diese Prüfungsordnung nicht in Kraft treten können.
\s{\Gandalf} Der AstA-Vorsitzende ist uns keine Hilfe. Er hat bereits unterzeichnet.
\s{\Elron} Nein.
\s{\Gandalf} Doch.
\s{{Chor}} Ohhhh!
\s{\Frodo} Kann ich sie nicht einfach kaputt machen? In den Müll werfen?
\s{\Gandalf} Nein. Sie kann nur da zerstört werden, wo sie geschaffen wurde. Wir müssen ins Rektorat.
\s{\Elron} Man geht nicht einfach ins Rektorat.
\s{\Gimli} Dann brauchen wir einen Plan. Und eine Gruppe wackerer Helden.
\s{\Elron} Ich werde von hier aus mit \Galadriel den Widerstand organisieren.
\s{\Frodo} Ich habe die Prüfungsordnung, ich werde sie auch vernichten. \kregie{bindet Prüfungsordnung an Kette}
\s{\Sum} YOLO! Ich bin dabei.
\s{\Gandalf} Auf mich werdet ihr nicht verzichten können, ich kenne alle Wege zum Rektorat.
\s{\Legolars} Du kannst mit meinem Bogen rechnen. \kregie{Hält Bogen hoch}
\s{\Gimli} Und mit meiner Axt! Und meinem Sandwich! \kregie{Hält erst Axt, dann auch Sandwich hoch}
\end{verseplay}

\regie{Licht aus.}

\newpage
\section{Band: Männer mit Bärten}
\label{sec:band_maenner}
\charaktere{\Chor \Sing}
\setting{Band}
\hauptbeamer{Text}
\sound{Band}
\licht{Band}
\requisiten{2 Keyboards, 3 Gitarren, 1 Bass, 1 Geige, 1 Flöte, 1 Schlagzeug, Mikros}


\begin{verseplay}[10em]
\s{\Sing} Alle die mit uns zur Hauptmensa gehen\\
Müssen Männer mit Bärten sein\\
Alle die mit uns zur Hauptmensa gehen\\
Müssen Männer mit Bärten sein\\
\s{\Chor}Jana, Klara, Kim, Judith, die haben Bärte, die haben Bärte\\
Jana, Klara, Kim, Judith, die haben Bärte, die gehen mit\\ 
\end{verseplay}
\begin{verseplay}[10em]


\s{\Sing} Alle die ANA und LA hören \\
Müssen Männer mit Bärten sein\\
Alle die ANA und LA hören \\
Müssen Männer mit Bärten sein\\
\s{\Chor}Jana, Klara, Kim, Judith, die haben Bärte, die haben Bärte\\
Jana, Klara, Kim, Judith, die haben Bärte, die hören mit\\ 

\end{verseplay}
\begin{verseplay}[10em]
\s{\Sing} Alle die Skripte und Bronnstein lesen\\
Müssen Männer mit Bärten sein\\
Alle die Skripte und Bronnstein lesen\\
Müssen Männer mit Bärten sein\\
\s{\Chor}Jana, Klara, Kim, Judith, die haben Bärte, die haben Bärte\\
Jana, Klara, Kim, Judith, die haben Bärte, die lesen mit\\ 
\end{verseplay}
\begin{verseplay}[10em]

\s{\Sing} Alle die mit uns das Übungsblatt lösen\\
Müssen Männer mit Bärten sein\\
Alle die mit uns das Übungsblatt lösen\\
Müssen Männer mit Bärten sein\\
\s{\Chor}Jana, Klara, Kim, Judith, die haben Bärte, die haben Bärte\\
Jana, Klara, Kim, Judith, die haben Bärte, die lösen mit\\ 
\end{verseplay}
\begin{verseplay}[10em]

\s{\Sing} Alle, die Prüfung und Vortrag nicht fürchten\\
Müssen Männer mit Bärten sein\\
Alle, die Prüfung und Vortrag nicht fürchten\\
Müssen Männer mit Bärten sein\\
\s{\Chor}Jana, Klara, Kim, Judith, die haben Bärte, die haben Bärte\\
Jana, Klara, Kim, Judith, die haben Bärte, die Schreiben mit\\ 
\end{verseplay}
\begin{verseplay}[10em]

\s{\Sing} Alle die Fettigen Mensafraß lieben\\
Müssen Männer mit Bärten sein\\
Alle die Fettigen Mensafraß lieben\\
Müssen Männer mit Bärten sein\\
\s{\Chor}Jana, Klara, Kim, Judith, die haben Bärte, die haben Bärte\\
Jana, Klara, Kim, Judith, die haben Bärte, die fressen mit\\ 
\end{verseplay}
\begin{verseplay}[10em]

\s{\Sing} Alle die mit uns zur Erstifahrt fahren\\
Müssen Männer mit Bärten sein\\
 Alle die mit uns zur Erstifahrt fahren\\
Müssen Männer mit Bärten sein\\
\s{\Chor}Jana, Klara, Kim, Judith, die haben Bärte, die haben Bärte\\
Jana, Klara, Kim, Judith, die haben Bärte, die fahren mit\\ 
\end{verseplay}
\begin{verseplay}[10em]

\s{\Sing} Alle die endlich den Master dann Haben\\
Müssen Männer mit Bärten sein\\
Alle die endlich den Master dann Haben\\
Müssen Männer mit Bärten sein\\
\s{\Chor}Jana, Klara, Kim, Judith, die haben Bärte, die haben Bärte\\
Jana, Klara, Kim, Judith, die haben Bärte, die gehen mit\\ 

\end{verseplay}

\regie{Licht aus.}

%Nach Moria

\newpage
\section{Nach Moria}
\label{sec:nach-moria}
    \charaktere{\Frodo, \Sum, \Gimli, \Legolars, \Gandalf}
    \setting{Mario-Hintergründe, bewegt}
    \hauptbeamer{nichts}
    \sound{nichts, nachher Supersterneffekt}
    \licht{Spot}
    \requisiten{Mario-Hintergründe}
    

\regie{Sie laufen durch den Wald. Sie laufen durch den Wald. Sie laufen durch den Wald. Sie laufen durch den Wald. Sie laufen durch den Wald.}
\begin{verseplay}[10em]


\s{\Sum} Sind wir schon da?
\s{\Gandalf} Nein.

\end{verseplay}
\regie{Werbepause, z.B. Mensa-Card, danach Mate, evtl. mehr}
\begin{verseplay}[10em]

\s{\Sum} Sind wir jetzt da?
\s{\Gandalf} Nein.
\end{verseplay}

\regie{Jetzt nur ein Spot, z.B. Klingeltöne}
\begin{verseplay}[10em]

\s{\Sum} Sind wir jetzt da?
\s{\Gandalf} Ja.
\s{\Sum} \kregie{Aufgeregt} Wirklich?
\s{\Alle} NEIN!

\s{\Legolars} Wo gehen wir eigentlich lang?
\s{\Gandalf} Über den großen Mensabrückenpass.
\s{\Gimli} Wir können auch bei meinem Vetter Moria vom Nerdpol durch den LAN-Partykeller gehen. Ein altes System aus Gängen führt von dort aus direkt zur Verwaltung.
\s{\Sum} Gibts da auch was zu essen?
\s{\Gimli} Dort wird uns ein Fest geboten, mit Tunefisch aus der Nerdsee.
\s{\Sum} Mhhhhhhmmm, lecker.
\s{\Gandalf} Dann auf zu Moria.
\end{verseplay}

\regie{Sie laufen durch den Wald. Gumbas und Kröten laufen vorbei. Sie springen drüber.}

\begin{verseplay}[10em]
\s{\Sum} Das dauert aber lange.
\end{verseplay}

\regie{Eine Fragezeichenbox erscheint. Gandalf schlägt seinen Laserstock dagegen. Ein Stern springt heraus. Frodo fängt ihn. Mario-Stern-Melodie, alle gehen sehr schnell. Nach einigen Sekunden abblenden, gleichzeitig Kartenvideo auf Beamer, dabei wird ein Tor aufgebaut. Blende zurück. Sie stehen erschöpft vor dem Tor.}

\begin{verseplay}[10em]
\s{\Gimli} Wir haben jetzt kurz nach drei. Um diese Zeit schläft mein Vetter eh noch. Lasst uns erst einmal rasten.
\s{\Sum} Aber ich will zum Festmahl!
\s{\Legolars} Gandalf, kannst du den Einbruch der Dunkelheit nicht beschleunigen?
\end{verseplay}
\regie{Gandalf stampf seinem Laserstock drei mal auf. Es wird langsam dunkel. UV-Licht geht langsam an. Captcha in elbischer Schrift auf der Tür erscheint.}

\begin{verseplay}[10em]
\s{\Frodo} Was ist denn da erschienen?
\s{\Gimli} Das ist das Einbruchsschutzcaptcha meines Vetters. Er schützt sich davor vor Wirtschaftswissenschaftlern.
\s{\Gandalf} \kregie{liest vor} Wer das liest ist doof.
\s{\Legolars} \kregie{tippt ein} Funktioniert nicht.
\s{\Gandalf} Klein und zusammen geschrieben.

\end{verseplay}

\regie{\Legolars tippt ein. UV aus. Tür öffnet sich.}


\newpage
\section{Werbung: Schreib dich nicht ab}
\label{sec:schreibdichnichtab}
\charaktere{-}
\setting{-}
\hauptbeamer{Werbungs-Intro, danach ``Schreib dich nicht ab''}
\sound{-}
\licht{-}
\requisiten{-}

\newpage
\section{Baum im Herbst}
\label{sec:baumimherbst}
\charaktere{\Baum}
\setting{-}
\hauptbeamer{-}
\sound{-}
\licht{erst -, dann Sport}
\requisiten{-}

\begin{verseplay}[7em]
\s{{Off}} Und nun ein Mann, der einen Baum im Herbst darstellt.
\end{verseplay}

\regie{Spot an, Baum im Herbst.}

\regie{Licht aus.}
\newpage
\section{Werbung: ESAG Mystery}
\label{sec:esagmystery}
\charaktere{-}
\setting{-}
\hauptbeamer{``ESAG-Mystery''}
\sound{-}
\licht{-}
\requisiten{-}

\newpage
\section{Band: Never gonna give you up}
\label{sec:band_never}
\charaktere{Band}
\setting{Band}
\hauptbeamer{-}
\sound{-}
\licht{Band}
\requisiten{-}

\regie{Die Band spielt ``Never gonna give you up''}

\regie{Licht aus.}

\newpage
\section{Werbung: Mensacard}
\label{sec:mensacard}
\charaktere{-}
\setting{-}
\hauptbeamer{``Mensacard'', danach Werbungs-Outro}
\sound{-}
\licht{-}
\requisiten{-}

\newpage
\section{Nach Moria 2}
\label{sec:nach-moria2}
\charaktere{\Frodo, \Sum, \Gimli, \Legolars, \Gandalf}
\setting{Mario-Hintergründe, bewegt}
\hauptbeamer{-}
\sound{-}
\licht{Spot}
\requisiten{Mario-Hintergründe}

\begin{verseplay}[7em]
\s{\Sum} Sind wir jetzt da?
\s{\Gandalf} Nein.
\end{verseplay}

\regie{Licht aus.}

\newpage
\section{Werbung: Klingeltöne}
\label{sec:klingelton}
\charaktere{-}
\setting{-}
\hauptbeamer{erst -, danach ``Klingeltöne''}
\sound{-}
\licht{-}
\requisiten{-}

\begin{verseplay}[7em]
\s{{Off}} Jetzt nur ein Spot.
\end{verseplay}

\regie{Hauptbeamer: ``Klingeltöne''}

\newpage
\section{Nach Moria 3}
\label{sec:nach-moria3}
\charaktere{\Frodo, \Sum, \Gimli, \Legolars, \Gandalf}
\setting{Mario-Hintergründe, bewegt}
\hauptbeamer{erst -, dann Kartenvideo}
\sound{-}
\licht{Spot}
\requisiten{Mario-Hintergründe, Tor}

\begin{verseplay}[7em]
\s{\Sum} Sind wir jetzt da?
\s{\Gandalf} Ja.
\s{\Sum} \kregie{aufgeregt} Wirklich?
\s{{Alle}} NEIN!
\s{\Legolars} Wo gehen wir eigentlich lang?
\s{\Gandalf} Über den großen Mensabrückenpass.
\s{\Gimli} Wir können auch bei meinem Vetter Moria vom Nerdpol durch den LAN-Partykeller gehen. Ein altes System aus Gängen führt von dort aus direkt zur Verwaltung.
\s{\Sum} Gibts da auch was zu essen?
\s{\Gimli} Dort wird uns ein Fest geboten, mit Tunefisch aus der Nerdsee.
\s{\Sum} Mhhhhhhmmm, lecker.
\s{\Gandalf} Dann auf zu Moria.
\end{verseplay}

\regie{Sie laufen durch den Wald. Gumbas und Kröten laufen vorbei. Sie springen drüber.}

\begin{verseplay}[7em]
\s{\Sum} Das dauert aber lange.
\end{verseplay}

\regie{Eine Fragezeichenbox erscheint. Gandalf schlägt seinen Laserstock dagegen. Ein Stern springt heraus. Frodo fängt ihn. Mario-Stern-Melodie, alle gehen sehr schnell. Nach einigen Sekunden abblenden, gleichzeitig Kartenvideo auf Beamer, dabei wird ein Tor aufgebaut. Blende zurück. Sie stehen erschöpft vor dem Tor.}

\begin{verseplay}[7em]
\s{\Gimli} Wir haben jetzt kurz nach drei. Um diese Zeit schläft mein Vetter eh noch. Lasst uns erst einmal rasten.
\s{\Sum} Aber ich will zum Festmahl!
\s{\Legolars} \Gandalf, kannst du den Einbruch der Dunkelheit nicht beschleunigen?
\end{verseplay}
\regie{\Gandalf stampft seinem Laserstock drei mal auf. Es wird langsam dunkel. UV-Licht geht langsam an; ein Captcha in elbischer Schrift auf der Tür erscheint.}

\begin{verseplay}[7em]
\s{\Frodo} Was ist denn da erschienen?
\s{\Gimli} Das ist das Einbruchsschutzcaptcha meines Vetters. Er schützt sich damit vor Wirtschaftswissenschaftlern.
\s{\Gandalf} \kregie{liest vor} Wer das liest ist doof.
\s{\Legolars} \kregie{tippt ein} Funktioniert nicht.
\s{\Gandalf} Klein und zusammen geschrieben.

\end{verseplay}

\regie{\Legolars tippt ein. UV aus. Tür öffnet sich. Licht aus.}

\newpage
\section{Moria}
\label{sec:moria}
\charaktere{\Frodo, \Sum, \Gimli, \Legolars, \Gandalf, \Monk, \Leichen}
\setting{Dunkle Höhle, Tische mit PCs und leblose Körper}
\hauptbeamer{-}
\sound{-}
\licht{erst Spot, dann normal}
\requisiten{Lichtschalter, Tische, PCs}
    
\regie{Monk sitzt die ganze Zeit in der Ecke und wippt ``Tuch, Tuch''. Die Gruppe kommt auf die Bühne. Spot auf die Gruppe, der Rest ist abgedunkelt.}

\begin{verseplay}[7em]
\s{\Sum} Mann, ist das hier dunkel. Man sieht seine eigenen Füße nicht.
\s{\Gandalf} Lumos! \kregie{schaltet seinen Stab ein und wedelt pottermäßig rum}
\s{\Gimli} Das geht auch einfacher. \kregie{schaltet großen Lichtschalter ein}
\end{verseplay}

\regie{Licht an. Alle erschaudern, untersuchen die leblosen Körper.}

\begin{verseplay}[7em]
\s{\Gandalf} \kregie{schaut von einer Leiche hoch} Das ist keine LAN-Party, das ist ein Grab!
\s{\Gimli} \kregie{fällt auf die Knie} Neeeeeiiiiiiin!
\s{\Frodo} Was ist hier passiert?
\s{\Sum} Sie sind tot, sie sind alle tot.
\end{verseplay}

\regie{\Sum stupst an den Leichen rum. Während des nächsten Gespräches mit \Monk setzt sich \Sum unauffällig an einen Rechner. \Frodo guckt ebenso unauffällig zu.}

\begin{verseplay}[7em]
\s{\Legolars} \kregie{entdeckt Monk, zeigt hin} Guckt mal, da hinten lebt noch einer!
\s{\Gandalf} \kregie{geht hin, klopft Monk auf die Schulter} Hallo?
\s{\Monk} \kregie{wischt hektisch seine Schulter} Nicht berühren! \kregie{hüpft weg}
\s{\Gandalf} Wer bist du? Was ist hier passiert?
\s{\Monk} Haben sie ein Tuch? Ich brauche ein Tuch.
\s{\Gandalf} \kregie{guckt sich suchend um}
\s{\Legolars} \kregie{kommt, gibt Tuch}
\s{\Monk} Vielen Dank! \kregie{säubert sich}
\s{\Monk} Mein Name ist Adrian Monk. Könnte ich noch ein Tuch bekommen?
\s{\Legolars} \kregie{gibt Tuch}
\s{\Monk} \kregie{säubert sich weiter} Danke. So wie ich das rekonstruieren kann, hat hier vor viereinhalb Wochen eine LAN-Party stattgefunden. Erst haben sie noch jeden Tag Pizza bestellt, aber dann waren sie so sehr in ihr Spiel vertieft, dass sie vergessen haben, zu essen.
\s{\Gimli} \kregie{seufzt und schluchzt stärker}
\s{\Legolars} Und was machen sie hier?
\s{\Monk} Ich war vorletzte Woche mit meiner Assistentin Natalie auf dem Weg zur Universitäts- und Landesbibliothek. Wir wollten eigentlich über den großen Mensabrückenpass gehen, aber dann wären wir nicht immer schön abwechselnd links und rechts abgebogen. Deshalb sind wir durch diesen Keller gegangen und haben die Leute hier so vorgefunden.
\s{\Gandalf} Das war vor zwei Wochen, was machst du immernoch hier und wo ist deine Assistentin?
\s{\Monk} Haben sie gesehen, wie das hier aussieht? Ich musste selbstverständlich erstmal hier aufräumen. Überall Pizzakartons, Tetra-Packs, nicht nach Geschmack sortiert, die Leichen lagen einfach verstreut in der Gegend herum, überall Staub und Fettflecken und man kann hier noch nicht mal lüften. Letzte Woche war es Natalie zu viel und sie ist aus unerfindlichen Gründen gegangen. Aber das Schlimmste war, dass mir vor drei Tagen die Taschentücher und Sagrotan ausgegangen sind. \kregie{zu \Legolars} Haben sie noch ein Tuch für mich?
\s{\Legolars} \kregie{reicht ihm ein Tuch}
\s{\Monk} \kregie{putzt weiter}
\s{\Sum} Hey, das Spiel ist echt gut!
\s{\Gandalf} \kregie{geht zu zerstörtem \Gimli} Wir müssen weiter, bevor einige von uns \kregie{zeigt mit dem Auge auf \Sum} auch dem Spiel verfallen.
\s{\Gimli} \kregie{nickt bedrückt und richtet sich auf}
\s{\Gandalf} \kregie{bestimmt} Wir gehen weiter!
\s{\Sum} \kregie{wehrt sich, wird von \Frodo mitgezogen}
\end{verseplay}

\regie{Alle ausser \Monk gehen langsam nach rechts ab, \Monk zögert, folgt monkig mit den Worten ``Tuch, Tuch, Tuch''. Licht aus.}

%Moria 2

\newpage
\section{Moria 2}
\label{sec:moria2}
    \charaktere{\Frodo, \Sum, \Gimli, \Legolars, \Gandalf, \Monk, \Pacman}
    \setting{Dunkle Höhle, Stellwände auf beiden Seiten}
    \hauptbeamer{nichts}
    \sound{nichts}
    \licht{}
  \requisiten{2 Stellwände}
    
\regie{Die Party betritt die Bühne von hinter Stellwand(a). Monk monkt rum.}

\begin{verseplay}[10em]
\s{\Gandalf} Ich spüre eine Erschütterung der Macht.
\s{\Sum} Vielleicht hast du einfach nur Hunger.
\s{\Gimli} \kregie{sarkastisch} Möchtest du mein Sandwich?
\s{\Frodo} Das ist wirklich lecker! Und fast frisch!
\s{\Legolars} \kregie{Die Idioten ignorierend} Ich spüre auch etwas. Gandalf?
\s{\Gandalf} \kregie{Kunstpause} Es ist ein Dämon aus der alten Zeit. \kregie{kurze Pause} Lauft!
\end{verseplay}

\regie{Pacman erscheint waka-waka-end aus der Stellwand(a) und jagt die Party hinter die Stellwand(b). Licht aus.}
%Moria 3

\newpage
\section{Moria 3}
\label{sec:moria3}
    \charaktere{\Frodo, \Sum, \Gimli, \Legolars, \Gandalf, \Monk, \Pacman}
    \setting{Dunkle Höhle, Stellwände auf beiden Seiten}
    \hauptbeamer{nichts}
    \sound{nichts}
    \licht{}
  \requisiten{2 Stellwände}
    
\regie{Die Party, Gandalf als letzter, hetzt auf die Bühne von hinter der Stellwand(a). Monk monkt rum. Gandalf hält auf Mitte der Bühne an, die anderen halten verzögert. Gandalf dreht sich zu Pacman. Pacman stoppt. Während Gandalfs Monolog weicht Pacman langsam zurück.}

\begin{verseplay}[10em]
\s{\Gandalf} Ich bin ein Diener der Wissenschaften und Gebieter über die Gesetze der Natur. Du - kannst - nicht - vorbei! \kregie{schlägt mit Stock auf den Boden}
\s{\Pacman} \kregie{Weicht kurz stärker zurück, taumelt. Sprintet waka-waka-end los.}
\s{\Gandalf} Flieht ihr Narren!
\end{verseplay}

\regie{Die Party flieht hinter Stellwand(b). Pacman frisst Gandalf, indem er vor ihn rennt und Gandalf sich klein macht. Licht aus.}

\begin{verseplay}[10em}
\s{\Frodo} Gandaaaaalf!
\end{verseplay}

\newpage
\section{Moria 4}
\label{sec:moria4}
\charaktere{\Frodo, \Sum, \Gimli, \Legolars, \Monk}
\setting{Dunkle Höhle, Stellwand, Star Gate}
\hauptbeamer{nichts}
\sound{nichts}
\licht{normal}
\requisiten{Stellwand(a), Star Gate}
    
\regie{Die Party läuft von hinter Stellwand(a) und stoppt in der Mitte. \Monk monkt rum.}

\begin{verseplay}[7em]
\s{\Legolars} Ich glaube wir sind hier erstmal in Sicherheit.
\end{verseplay}

\regie{\Frodo und \Sum hocken sich hin und tuscheln bedröppelt.}

\begin{verseplay}[7em]
\s{\Gimli} \kregie{guckt sich um, entdeckt das Star Gate} Guck mal \Legolars, ich glaube hier gehts raus!
\s{\Legolars} \kregie{geht hin, untersucht das Star Gate, kurze Pause, leicht überrascht} Das ist ein Star Gate.
\s{\Gimli} Kannst du es bedienen?
\s{\Legolars} \kregie{dreht das Glücksrad} Sieht gut aus. \Gimli, \Sum, geht ihr als erstes.
\end{verseplay}

\regie{\Sum und \Frodo stehen auf. \Gimli und \Sum gehen durch das Star Gate. \Monk monkt am Gerät rum.}

\begin{verseplay}[7em]
\s{\Monk} Das ist ja total verstaubt. \kregie{entstaubt Glücksrad, Glücksrad verstellt sich}
\s{\Legolars} Neiiiiin, was hast du getan?
\s{\Monk} Es war dreckig.
\s{\Legolars} Du hast die Adresse verstellt! Egal, wir müssen hier weg. Kommt!
\end{verseplay}

\regie{\Frodo, \Monk und \Legolars gehen durch das Star Gate. Licht aus.}

%
%        BEGIN: QED
%
\newpage
\section{QED - der Shopping-Sender}
\label{sec:QED}
        \charaktere{\QEDHost, \QEDGuestA, \QEDGuestB \QEDGuestC}
        \setting{3 Tische auf Seite der B"uhne}
        \hauptbeamer{}
        \sound{}
        \licht{}
        \requisiten{3 Tische, Bierflasche(voll), Steinschleuder, Taschenlampe, 2xLaptop, Stapel Papier, Rechenschieber, DVD-H"ullen, Buch mit H"ulle, K"uhlpack, Diskette}
\regie{Gastgeber und erster Gast treten auf.}
\begin{verseplay}[5em]
\s{\QEDHost} Herzlich willkommen zur Dauerwerbesendung auf QED. Ich begr"u"se 
                Sie zum Tag der Naturwissenschaft, bei dem wir Ihnen Waren der
                f"uhrenden Naturwissenschaften zu unschlagbaren Preisen anbeiten 
                werden. Sie erreichen uns die ganze Sendun "uber unter der 
                Telefonnummer 0211-81-3,1415926535. Nat"urlich k"onnen Sie diese 
                Nummer auch mit 3 ann"ahern.\\
                \kregie{zum ersten Gast}\\
                Fangen wir an bei der genialen Informatik. \QEDGuestA, was hast
                du uns denn heute Legend"ares mitgebracht?
\s{\QEDGuestA} Ja, \QEDHost. Also,ich habe dir heute etwas mitgebracht, 
                wovon die meisten Menschen in den B"uros dieser Welt schon immer 
                getr"aumt haben. Es verbindet nicht einen Computer, nicht zwei, 
                \dots, nicht f"unf, nein, es verbindet alle Computer miteinander, 
                die man sich nur vorstellen kann. Ich habe dir heute ein 
                100-Meter-W-LAN-Kabel mitgebracht!
\s{\QEDHost} Oh mein Gott. Das ist soooo neu. Das ist ja unglaublich! Ich 
                kann mich kaum halten vor Begeisterung. Erz"ahl uns mehr, \QEDGuestA.
\s{\QEDGuestA} Ich kann deine Begeisterung verstehen, \QEDHost. Mir ging es 
                genauso, als ich zum ersten Mal live erleben durfte, wie sich 
                meine Computer miteinander verbunden haben. Ich f"uhre das einmal 
                vor.\\
                \kregie{steckt unsichtbares Kabel in zwei Laptops}\\
                Siehst du, \QEDHost, wie die Dateien wie von Geisterhand 
                "ubertragen werden?
\s{\QEDHost} Oh ja, \QEDGuestA. Ich sehe es. Leider, liebe Zuschauer, 
                k"onnen Sie dieses ph"anomenale Ereignis nicht mit erleben, aber es 
                ist toll!
\s{\QEDGuestA} Und so schnell! Aber das ist noch nicht alles! Als 
                Dankesch"on f"ur Ihren Einkauf schenken wir Ihnen noch den 
                aktuellsten universal-Mousepad-Treiber f"ur alle Betriebssysteme, 
                und - als w"are das nicht schon genug - noch 50 blanko PDF-Dateien 
                obendrauf!
\s{\QEDHost} Oh mein Gott, das ist ja unglaublich. Kaum zu fassen, dass es 
                solche Angebote zu diesem Preis gibt!
\s{\QEDGuestA} Ja, und f"ur die ganz schnellen Anrufer gibt es sogar 75 
                Meter Kabel gratis dazu! 
\s{\QEDHost} Ja, aber \QEDGuestA, jetzt verrate uns auch den Preis.
\s{\QEDGuestA} Gerne, \QEDHost. Dieses Angebot kann schon f"ur unglaublich 
                g"unstige $150$ Euro bestellen. Laut Wikipedia ist das weniger als 
                ein Euro pro Meter!
\s{\QEDHost} Das ist ja kaum zu glauben! Liebe Zuschauer, was f"ur eine 
                n"utzliche Erfindung. Bestellen Sie jetzt diesen Artikel unter der 
                Nummer 0211-81-3,1415926535 mit dem Stichwort geek.
                Vielen Dank, \QEDGuestA.\\
                \kregie{\QEDGuestA tritt ab, \QEDGuestB tritt auf}\\
                Kommen wir nun zur grandiosen Physik. Hallo \QEDGuestB. Was hast 
                du uns heute denn Unglaubliches mitgebracht?
\s{\QEDGuestB} Hallo \QEDHost. Ich pr"asentiere dir heute einen 
                ultra-kompakten Teilchenbeschleuniger: die Teilchenschleuder 
                XI-1300-A.
\s{\QEDHost} Wow, was f"ur eine tolle Sache!
\s{\QEDGuestB} Ja, ich demonstriere direkt einmal ihre  Wirkung. Du nimmst 
                ein beliebiges Teilchen und f"ugst es in die "Offnung der 
                Teilchenschleuder K-807-V ein. Dann spannst du es ganz einfach 
                leicht und zielst es gegen beliebige Ziele.\\
                \kregie{schleudert Kreide gegen Tafel}
\s{\QEDHost} Das ist ja unglaublich! Woe einfach das aussieht!
\s{\QEDGuestB} Es sieht nicht nur einfach aus,es ist einfach!
\s{\QEDHost} Das ist ja legend"ar,  aber doch bestimmt nicht alles, was du 
                uns mitgebracht hast, oder \QEDGuestB?
\s{\QEDGuestB} Stimmt, \QEDHost, denn wer jetzt sofort anruft, bekommt 
                diese praktische K"altepumpe gratis dazu! Mit ihr kannst du deine 
                Teilchen auf bis zu minus zehn Kelvin herunterk"uhlen!
\s{\QEDHost} Oh mein Gott! Das ist ja unglaublich! Dann kann man ja selber 
                Schneeb"alle herstellen und diese mit der Teilchenschleuder Z-654-J 
                beschleunigen. Welch ein Spa"s f"ur Kinder!
\s{\QEDGuestB} Und f"ur die ersten $10.000$ Anrufer gibt es auch noch 
                kostenlos diese limitierte, extrem handliche Photonenkanone dazu. 
                All diese Angebote kosten nur $13,37$ Euro plusminus 10 Prozent!
\s{\QEDHost} Oh mein Gott, das ist unglaublich! Vielen Dank f"ur diese 
                praktischen Gegenst"ande. Liebe Zuschauer, bestellen Sie jetzt 
                unter der 0211-81-3,1415926535 mit dem Stichwort Quark!
                \kregie{\QEDGuestB tritt ab, \QEDGuestC tritt auf}\\
                Und last but not least stelle ich Ihnen Neuheiten aus der Welt der 
                Mathematik vor. Hallo \QEDGuestC, was hast du uns denn mitgebracht?
\s{\QEDGuestC} Hallo \QEDHost, ich habe f"ur unsere Zuschauer eine massive 
                Erleichterung des mathematischen Alltags mitgebracht: nicht eine,
                nicht zwei, nein tausend unentdeckte Primzahlen. Diese Primzahlen 
                werden von f"uhrenden Mathematikern schon lange Zeit gesucht und 
                nun haben sie zu Hause, liebe Zuschauer, die M"oglichkeit, dieser 
                Suche ein Ende zu bereiten, und das f"ur nur $0,1995\cdot10^2$ Euro. 
\s{\QEDHost} Wahnsinn! Haben Sie das geh"ort, liebe Zuschauer! Schlagen sie 
                jetzt zu und greifen Sie sofort zum H"orer.
\s{\QEDGuestC} Aber das ist noch nicht alles. Damit Sie auch neben den 1000 
                unentdeckten primzahlen viel Freude haben, schenken wir ihnen 
                gratis noch den durch mehrere Preise geehrten Film "`one night im 
                hilbertraum"' als Doppel-DVD mit dem "`Schweigen der Lemmata"' und 
                den legend"aren Stummfilm "`Die letzte Stelle von Pi"' dazu. 
\s{\QEDHost} Oh mein gott, das ist ja unglaublich!
\s{\QEDGuestC} Ja, \QEDHost. Dieser tolle Stummfilm wird von Pianomusik des 
                ber"uhmten Pianisten G. Au"s untermalt. Und jetzt kommts: F"ur die 
                ersten 100 Anrufer gibt es noch einen Rechenschieber, mit dem es 
                m"oglich ist durch null zu teilen.
\s{\QEDHost} Das ist so unglaublich. \QEDGuestC, ich glaube da kann keiner 
                der Zuschauer widerstehen. Vielen dank f"ur diese tolle 
                Pr"asentation. Bestellen sie jetzt unter der Nummer 
                0211-81-3,1415926535 mit dem Stichwort Pi-Pi.
                \kregie{kurze Pause}
                Ich bin "uberw"altigt von all diesen Produkten und bin mir sicher, 
                dass Sie genauso empfinden. Auf Wiedersehen!
\end{verseplay}
%
%        END: QED
%Vorlesung

\newpage
\section{Vorlesung}
\label{sec:vorlesung}
    \charaktere{\Sum, \Gimli, \Prof, \Studa, \Studb, \Studc}
    \setting{Leere Tafel. ``Stargate'' Möglichst neben der ``Professorentafel'' Die Stühle Stehen auf der Anderen Seite als ``Seminaraumsetting''}
    \hauptbeamer{nichts}
    \sound{nichts}
    \licht{Bitte}
  \requisiten{Stargate, Fünf Stühle, Pult, Kreide}
\regie{\Prof steht an der Tafel \Studa, \Studb, \Studc Sitzen auf den Stühlen. \Gimli und \Sum sind hinter dem Stargate versteckt.}    

\begin{verseplay}[3em]
\s{\Prof} Hallo liebe Studenten. Im heutigen Seminar zu den Komperativen Vergleichswissenschaften. Wie ich in der letzten Vorlesung schon angedeutet hatte, möchten wir uns heute mit überbegrifflichkeiten beschäftigen.
\end{verseplay}
\regie{geht zur Tafel und zeichnet an "Ü"}

\begin{verseplay}[3em]
 \s{\Prof} Wenn man nicht ins Detail gehen kann, sind Überbegrifflichkeiten Notwendig. So ist z.b. das "Dingsgedingsel" eine Überbegrifflichkeit, die alles bezeichnen kann, was bei Fünf nicht auf den Bäumen ist. Aber hier sehen wir auch schon die ersten Grenzen. Denn, möchte man in der Zoologie der Eichhörnchen und anderer Baumbewohner ungenau sein ist dieses Wort schlecht einzusetzen. Man kann "Dingsgedingsel" in den Vergleichenden Baumsäugerwissenschaften geradezu als Synonym für Faultiere nutzen.
\end{verseplay}
\regie{Auftritt \Sum und \Gimli durch das Stargate}
\begin{verseplay}[3em]
\s\Prof Wenn sie Schon zu Spät kommen können sie wenigstens so Freundlich sein, das Stargate im Hinteren Raum zu Nutzen. Setzen sie sich jetzt Bitte hin.
\s\Gimli \kregie{erschrocken} Wo zum Teufel sind wir hier?
\s\Prof Fragen beantworte ich gerne nach dem Seminar.
\s\Gimli Aber wir wollten doch nur \kregie{wird unterbrochen} ins Rekorat.
\s\Prof \kregie{nachdrücklich} Setzen sie sich bitte.
\end{verseplay}
\regie{\Sum und \Gimli lassen sich Einschüchtern und Setzen sich auf die Freien Plätze}

\begin{verseplay}[3em]
\s\Prof Wo war ich stehen geblieben?
\s\Studa Beim Gedingseldings
\s\Studb Er meint Dingsgedingsel, glaube ich
\s\Prof Ah, sehr Gut. Wie sie an diesem Beispiel sehen, gibt es bei Umschreibungsworten einige Fallen. Nehmen wir zum Beispiel das Dingsbumms.
\end{verseplay}
\regie{Geht zur Tafel und Schreibt Dingsbumms an.}


\begin{verseplay}[3em]
\s\Prof \kregie{wähend er Schreibt} Das Dingsbumms ist eine Übergeordnete Sprachkonstruktion, die so ziemlich alles Bedeuten kann währendessen das Bummsdings \kregie{schreibt Bummsdings daneben} in seiner Beschreibungsvariabilität deutlich eingeschränkter ist. Kann mir jemand ein Beispiel für ein Dingsbumms nennen?

\s\Studc Eine Melone \kregie{\Prof schreibt das unter Dingsbumms}
\s\Studa Eine Kalte Ente \kregie{\Prof schreibt das unter Dingsbumms}
\s\Studb Ein Motorrad \kregie{\Prof schreibt das unter Dingsbumms}
\s\Prof Genau. Wobei Das Motorrad auch unter Bummsdings fallen kann, wenn die Zündkerze nicht richtig eingestellt ist. Können sie sich auch etwas unter Bummsdings vorstellen?
\s\Studa Knaller \kregie{\Prof schreibt das unter Bummsdings}
\s\Studb Prellbock \kregie{\Prof schreibt das unter Bummsdings}
\s\Gimli Dynamit \kregie{\Prof schreibt das unter Bummsdings}
\s\Prof Sehr Gut. Dingsbummse können also jedwede Art von Ding sein, während ein Bummsdings ein Ding sein muss das entweder bummst, das gebummst wird oder das bummsen oder gebummst werden kann. 
\s\Studb Herr Professor?
\s\Prof Ja Bitte
\s\Studb Sehe ich das Richtig, das eine Knallerbse ein Bummsdings ist? Schliesslich ist der Knaller ja auch ein Bummsdings
\s\Prof \kregie{überlegt Kurz} Die Knallerbse ist tatsächlich kein Bummsdings. Das Knallen des Knallers lässt sich zwar als ``Bumms'' umschreiben, das Knallen einer Knallerbse geht jedoch eher in richtung Knall.
\s\Studb Dann ist also Poppkorn kein Bumsdings weil es ja Gepoppt aber nicht Gebummst wird.
\s\Prof Richtig. Auch wenn Poppen und Bumsen zur gleichen Geräuschfamilie gehören, wie übrigens auch das Knattern, muss beim Bummsdings hier genau unterschieden werden. Hat noch jemand Fragen?
\s\Studa Im letzten Übungsblatt, da wo wir Äpfel mit Birnen Vergleichen sollten kommt bei mir immer Ernte raus.
\s\Prof oh, da sollte eigentlich Ente herauskommen \kregie{geht hin und fängt an zu lesen} Das nehm ich eben mit ins Büro und geb ihnen danach Bescheid, vielleicht hat mein Assistent einen Fehler gemacht.
\s\Studa \kregie{zu \Gimli und \Sum} Kommt ihr noch mit auf ein Bier?
\s\Gimli Aber wir müssen dringend zum \kregie{wird unterbrochen} Rektorat.
\s\Sum Och komm, ein Bier geht doch...
\s\Studb Super, kommt.

\end{verseplay}

\regie{Licht aus.}


\newpage
\section{Band: Du hast}
\label{sec:band_duhast}
\charaktere{\Chor \Sing}
\setting{Band}
\hauptbeamer{Text}
\sound{Band}
\licht{Band}
\requisiten{Band}

\begin{verseplay}[10em]
\s{\Sing} Du\\
Du hast\\
Du hast mich\\
\end{verseplay}
\begin{verseplay}[10em]
\s{\Sing} Du\\
Du hast\\
Du hast mich\\
\end{verseplay}
\begin{verseplay}[10em]
\s{\Sing} Du\\
Du hast\\
Du hast mich\\
\end{verseplay}
\begin{verseplay}[10em]
\s{\Sing} Du\\
Du hast\\
Du hast mich\\
\end{verseplay}
\begin{verseplay}[10em]
\s{\Sing} Du\\
Du hast\\
Du hast mich\\
Du hast mich\\
Du hast mich gefragt\\
Du hast mich gefragt\\
Du hast mich gefragt und ich hab nichts gesagt\\
\end{verseplay}
\begin{verseplay}[10em]
\s{\Sing} Hast du denn die Übungszettel selbst gemacht ein einzges mal?\\
\s{\Chor}``aaaaaaah`` \s{\Sing} Nein\\
\s{\Chor}``aaaaaaah`` \s{\Sing} Nein\\
\end{verseplay}
\begin{verseplay}[10em]
\s{\Sing} Hast du denn die Übungszettel selbst gemacht ein einzges mal?\\
\s{\Chor}``aaaaaaah`` \s{\Sing} Nein\\
\s{\Chor}``aaaaaaah`` \s{\Sing} Nein\\
\end{verseplay}
\begin{verseplay}[10em]
\s{\Sing} Du\\
Du hast\\
Du hast mich\\
\end{verseplay}
\begin{verseplay}[10em]
\s{\Sing} Du\\
Du hast\\
Du hast mich\\
\end{verseplay}
\begin{verseplay}[10em]
\s{\Sing} Du\\
Du hast\\
Du hast mich\\
Du hast mich\\
Du hast mich gefragt\\
Du hast mich gefragt\\
Du hast mich gefragt und ich hab nichts gesagt\\
\end{verseplay}
\begin{verseplay}[10em]
\s{\Sing}Hast du denn die Übungszettel selbst gemacht ein einzges mal?
\s{\Chor}``aaaaaaah`` \s{\Sing} Nein\\
\s{\Chor}``aaaaaaah`` \s{\Sing} Nein\\
\end{verseplay}
\begin{verseplay}[10em]
\s{\Sing}Hast du in das Skript zur Prüfung reingeschaut ein einzges mal?
\s{\Chor}``aaaaaaah`` \s{\Sing} Nein\\
\s{\Chor}``aaaaaaah`` \s{\Sing} Nein\\
\end{verseplay}
\begin{verseplay}[10em]
\s{\Sing}Hast du denn die Übungszettel selbst gemaaaaaaaaacht
\s{\Chor}``aaaaaaah`` \s{\Sing} Nein\\
\s{\Chor}``aaaaaaah`` \s{\Sing} Nein\\

\end{verseplay}

\regie{Licht aus.}

\newpage
\section{Gandalf kehrt zurück}
\label{sec:gandalfkehrtzurueck}
\charaktere{\Gandalf, \Petrus, \Euler, \Waschmaschine}
\setting{Himmel(stor)}
\hauptbeamer{nichts}
\sound{leiser Engelsgesang?}
\licht{bitte gern}
\requisiten{Stranke/Absperrband, Waschmaschine, Buch, Riesenwuerfel, Waschmittel, Extraleben, Tisch, Stuhl, Nummernziehding, Numemernanzeigeding, Tuete/Begruessungspaket, jede Menge Karten (Mensakarten, Kreditkarten oder sowas), Federn/Engelsfl''ugel oder sowas, weisser Umhang fuer \Gandalf}

    
\regie{\Gandalf kommt in den Himmel Bühne, guckt verdutzt. \Petrus sitzt am Tisch}
\begin{verseplay}[5em]
\s{\Gandalf} Wo bin ich? Das ist nicht der Keller. Wo ist Pacman? \kregie{geht zu \Petrus} Entschuldigen Sie, wo \kregie{wird unterbrochen} bin ich hier?
\s{\Petrus} Nummer.
\s{\Gandalf} Was? Wo bin ich hier? Ich muss zurück zu \kregie{wird unterbrochen} meinen
\s{\Petrus} Nummer. Sie müssen eine Wartenummer ziehen \kregie{zeigt auf Nummernziehding}.
\s{\Gandalf} Aber \kregie{wird unterbrochen} ich muss
\s{\Petrus} Ohne Nummer kann ich Ihnen nicht weiterhelfen.
\end{verseplay}
\regie{\Gandalf geht zum Nummernziehding und zieht grummelnd eine Nummer.}
\begin{verseplay}[5em]
\s{\Gandalf} Das ist ja schlimmer als im Prüfungsamt hier.
\end{verseplay}
\regie{\Petrus ruft/zeigt die Nummer. \Gandalf geht zu \Petrus}
\begin{verseplay}[5em]
\s{\Gandalf} So, hier ihre Nummer. Ich muss jetzt echt \kregie{wird unterbrochen} zurück
\s{\Petrus} Name.
\s{\Gandalf} Wirklich? Es geht hier um Leben und Tod!
\s{\Petrus} Eher Tod. Sie sind hier im Himmel. Also: Ihr Name?
\s{\Gandalf} Tod? Nein, das kann nicht sein. Das DARF nicht sein! Ich muss sofort zurück! Wir müssen die Prüfungsordnung zerstören!
\s{\Petrus} Sie können nicht einfach so zurück wie es ihnen gefällt. Das darf nur der Sohn vom Chef. Sie können höchstens eine Ausnahmeurlaubserlaubnis beantragen. Poltergeist, gute Fee oder sowas in der Art, aaaaber dafür muss ich sie erst registrieren, also: ihr Name?
\s{\Gandalf} Nagut, wenns nicht anders geht. "`der Graue"'
\s{\Petrus} \kregie{guckt im Buch nach} "`der Alte"', "`der Dicke"', "`der Mathematiker"', ach ne zu weit. Ach hier "`der Graue"', Vorname \Gandalf, Tod durch gefressenwerden durch ein Monster, ist das korrekt?
\s{\Gandalf} Ja.
\s{\Petrus} Ok, \kregie{schreibt was in das buch} dann hier ihr Begrüungspaket \kregie{reicht \Gandalf Tüte}. Enthalten eine Karte der Umgebung, mit Öffnungszeiten unserer Bib, ein Regelbuch und ein paar Gutscheinen. Und hier \kregie{gibt riesenstapel an Karten} ihre Mensakarte, der Bibausweis, die Kopierkarte falls sie Bibelseiten kopieren wollen, ihr neuer Ausweis, Guthabenkarte für das Cafe, Ticket für den Wolkenexpress BLABLABLA UND JEDE MENGE ANDERE SACHEN.
\s{\Gandalf} Hätte man das nicht alles auf eine Karte packen können?
\s{\Petrus} Bestimmt, aber das wäre zu einfach gewesen, oder? So, jetzt würfeln Sie nurnoch aus was für einen Job Sie bekommen Herr "`Graue"' \kregie{greift würfel} und dann sind sie auch schon fertig.
\s{\Gandalf} Job? Ich dachte das wäre der Himmel? Und ich will hier eh so schnell wie es geht weg.
\s{\Petrus} Ja genau: Himmel, nicht 5-Sterne-Deluxe-Hotel. Denken Sie ich sitz hier seit 2000 Jahren vor der Himmelspforte und erkläre Leuten wie Ihnen die Regeln hier zum Spaß? Sie werden \kregie{\Gandalf würfelt und \Petrus guckt im Buch nach} ähm 17, das ist.... Waschküche. Sie werden in der Waschküche arbeiten. \Euler wird ihnen den Weg zeigen und ihnen den Rest erkläeren. \kregie{dreht sich um und ruft} \Euler!
\end{verseplay}
\regie{\Euler kommt (also zu den anderen, nicht auf der Bühne}
\begin{verseplay}[5em]
\s{\Euler} Ja Petrus, was gibts?
\s{\Petrus} Das ist Herr "`Graue"', zeig ihm bitte die Waschküche.
\s{\Euler} Klar, no Problemo. Na, dann komm mal mit
\end{verseplay}
\regie{\Euler und \Gandalf gehen auf die andere Seite der Bühne, dabei wird geredet}
\begin{verseplay}[5em]
\s{\Euler} So, du bist also der Neue. Mein Name ist VORNAME VON EULER.
\s{\Gandalf} Tach, ich bin \Gandalf. Was tust du hier so?
\s{\Euler} Ich helf manchmal Petrus mit den neuen Erstis, aber eigentlich bin ich für die Wolken zuständig und pass auf dass es immer eine prime Anzahl an Wolken gibt.\kregie{kommen an die Waschmaschine} So, da wären wir auch schon. Deine Aufgabe ist es die getragenen Engelsflügel zu waschen. Ist am Anfang ein bisschen eklig, aber wenn du gut arbeitest wirst du vielleicht in 213 Jahren oder so in die Sockenabteilung versetzt. Einfach Flügel rein, Waschpulver hinzu (WENN DER WITZ NOCH REINKOMMT: ABER NUR DAS GUTE VON /TOLLE FIRMA) und auf den Schleudergang. Viel Spaß.
\s{\Gandalf} Wo kann ich denn diese Auf-die-Erde-kommen-Anträge stellen?
\s{\Euler} Wolke 667, aber die ist mehr als böse, das dauert schonmal 37-43 Jahre bis man da einen Termin bekommt. Am besten machst du erstmal deine Arbeit hier, die öffnet eh erst erst in 7 Wochen. Wenn du noch fragen hast, in deiner Willkommestüte sit eine Miniharfe, damit kannst du mich anrufen.\kregie{geht ab (also weg, nicht tanzen!)}
\end{verseplay}
\regie{\Gandalf stopft die Flügel in die Waschmaschine, tut, Waschmittel hinzu, will starten}
\begin{verseplay}[5em]
\s{\Waschmaschine} Hallo, ich bin ihre vollautomatische hImmelswaschmine. Es wird mir eine Freude sein ihre dreckige Wäsche für Sie zu säubern.
\s{\Gandalf} Hmm, wenn ich hier eh warten muss, kann ich ja eigentlich auch meine eigenen Sachen waschen
\end{verseplay}
\regie{\Gandalf zieht sich aus und wirft seinen Mantel in die Waschmaschine. Entweder ist er jetzt nackt (wenn seine Haare (also die im Bart im Gesicht) lang genug sind, dann sieht man auch ncihts) oder er hat ein lustiges T-Shirt an (Pink, schwarz-gelb gestreift oder sowas). Dann stellt er die Maschine an. der Maschinenmotortyp sitzt in der Kiste und mach Lärm und wackelt. Plötzlich fällt das Extraleben aus \Gandalf s Bart auf den Boden}
\begin{verseplay}[5em]
\s{\Gandalf} Was ist das? Oh, das hatte ich ja total vergessen. Das Extraleben vom Schrottwichteln! Damit muss ich doch auf jeden Fall zurück nach Mitellerde/zur Uni /zur Erde /zu deiner Mudda kommen.
\s{\Waschmaschine} Hallo, hier spricht ihre vollautomatische Himmelswaschmaschine. Es bereitet mir große Freude Ihnen mitteilen zu können dass ich ihre dreckige Wäsche erfolgreich für sie säubern konnte.
\s{\Gandalf} Ah, das passt ja \kregie{nimmt sachen raus, guckt sich seinen WEISSEN mantel an} Oho, diese WITZ Waschmittel ist ja wirklcih klasse. Mein Mantel war noch nie so sauber! Jetzt schnell zu \Petrus \kregie{zieht mantel an und geht zu \Petrus}
\s{\Gandalf} So, hier \kregie{knallt EL auf den tisch}. Ich hab noch ein Extraleben. Damit werde ich doch wohl irgendwie zur MUDDA zurückkehren können, oder?
\s{\Petrus} Ach sie haben noch ein Leben. Wieso haben sie das nciht gleich gesagt? Natürclich. Einfach aktivieren und schon kehren sie automatisch zurück
\s{\Gandalf} öhm aktivieren? wie aktiviert man sowas denn? draufdrücken?
\s{\Petrus} also eigentlich sollten sie sowas ja wissen. EInfach das herz hochhalten und ihren persönlichen Catchphrase aufsagen.
\s{\Gandalf} catchphrase? hmm, sowas hab ich nicht. mal überlegen...\kregie{kurze pause, wartemusik}Oh ich habs! \kregie{hält Herz nach oben (wie link)} Ich bin nicht der Zauberer/Dozent/Juniorprofessor/Langzeitstudent den Mitellerde/die Uni /die Erde /deine Mudda verdient, ich bin der Zauberer/Dozent/Juniorprofessor/Langzeitstudent den Mitellerde/die Uni /die Erde /deine Mudda braucht
\end{verseplay}
\regie{Blitze, hell/dunkel/gewitter licht und soundeffekte, nackte frauen laufen über die bühne und ein panda auf einen einrad spielt den flowalzer auf einer trompete. lciht aus. sketch zuende, affe tod}


%Reimschiff

\newpage
\section{Reimschiff}
\label{sec:reimschiff}
    \charaktere{\Frodo, \Legolars, \Monk, \Paul, \Spock, \Pille}
    \setting{Raumschiff-Setting. Tisch-Kommandostuhl-Tisch, Stargate}
    \hauptbeamer{Sterne (Stardings?)}
    \sound{nichts}
    \licht{Bitte}
  \requisiten{Captainsstuhl, 2 Tische (Steuerkonsolen), Stargate}
\regie{Anwesend ist \Paul (auf Stuhl) \Spock (An der einen Konsole) und \Pille (An der anderen Konsole) Blickrichtung ist immer in Richtung.}    

\begin{verseplay}[10em]
\s{\Spock} Ich höre grad im Hyperraum\\
ein Lied, dass wie aus einem Traum\\
Nun folgt mein Rat in dieser Sache\\
dass man ne Überprüfung mache.

\s{\Paul} \kregie{Drückt an seiner ``Stuhlkonsole'' rum} Ich wiederspreche euch nicht gerne\\
doch Hyperraum ist weit gefehlt\\
es kommt vom runden Tor der Sterne\\
das euer Ohr mit Tönen quält

\s{\Spock} Ach lasst uns dennoch danach sehn\\
Denn sonst könnt es uns schlecht ergehn
\s{\Paul} Wohlann dann lasst uns tapfer schaun\\
Was kommt da aus dem Hyperraum
\end{verseplay}

\regie{\Pille Drückt irgendwas auf seiner Konsole. Stargate-Sound erschallt und \Legolars,\Frodo und \Monk erscheinen auf der Bildfläche. Monk monkt tonlos - aber dabei reimend - herum.}

\begin{verseplay}[10em]
\s{\Legolars} Oh Gott, wo sind wir denn gestrandet,\\
wer sind diese Leute hier?\\
warum sind wir nicht im Rektorat gelandet,\\
erklärt das jetzt mal jemand mir?
\s{\Monk} Hör bitte auf, dein Versmaß ist eine Qual...
\s{\Frodo} ...ich glaube, wir haben keine Wahl.\\
Warum müssen wir denn hier reimen?\\
Will uns hier jemand leimen?

\s{\Paul} Ihr seid auf einem Reimschiff hier\\
Doch Frage ich: wer seid denn ihr?\\
Das Tor, das euch zu uns gebracht\\
War nutzlos seit Walpurgisnacht\\
Als Jener dort mit bloßer Hand \kregie{Zeigt auf \Pille}\\
Den Supraleiter hat Verbrannt.\\
Und nun, als sei dies nicht geschehn\\
Drei Leute aus dem Tor rausgehn\\
Drumm sagt mir nun, ich wüsst es gern\\
Kommt ihr von einem andren Stern?

\s{\Legolars} Von drauss vom Walde komm' wir her,\\
der Weg hier hin war meiner Meinung nach auch sehr schwer,\\
zum Rektorat, da wolln wir hin,\\
die Rahmenprüfungsordnung zu vernichten ist der Sinn.

\s{\Monk} \kregie{krümmt sich.}

\s{\Spock} Das Rektorat, ein Böser Ort\\
Dort lässt man lebend keinen Fort\\
Es ist Des Rektors Dunkles Reich\\
Ich mach mir in die Hose Gleich\\
Der Ort der Ist Zu Stark bewacht\\
und Reimschiffsicher Tag und Nacht\\
Denn dort ein Jeder Reim versagt\\
Wo Rektor und Verwaltung Tagt

\s{\Frodo} Einer ist schon niedergegangen,\\
wir sind auf einem Reimschiff gefangen,\\
wir wissen nicht wohin wir wollen,\\
so kommt die Mission nicht ins rollen,\\
ob wir nicht einfach aufgeben sollen.

\s{\Paul} Ach gebt die Sache nicht verloren\\
Ein fünkchen Hoffnung mag noch sein\\
Ihr stoßt hier nicht auf Taube Ohren\\
Jedoch die Chance ist sehr Klein\\
Wir schaffe Tief in Unis Keller\\
Euch weit noch reins ins Klinikum\\
Von da aus is der Weg viel schneller\\
Dreht ihr euch nach dem Beamen Um.

\s{\Frodo} Vielen Dank ihr netten Leute,\\
dass ihr uns hier jetzt helft noch heute,\\
Mr. Monk hier wird schon bleich,\\
beeilen wir uns lieber gleich.

\s{\Paul} Dann lasst keine Zeit verlieren\\
doch ruhe für das manövrieren\\
denn eng und dunkel ist der Weg\\
den ich nun auf den Hauptschirm Leg 

\end{verseplay}

\regie{Licht aus.}

\newpage
\section{Band: Shift}
\label{sec:band_shift}
\charaktere{Band}
\setting{Band}
\hauptbeamer{-}
\sound{-}
\licht{Band}
\requisiten{-}

\regie{Die Band spielt ``Shift''}

\regie{Licht aus.}

\newpage
\section{Werbung: Club Mate}
\label{sec:clubmate}
\charaktere{-}
\setting{-}
\hauptbeamer{``Club Mate''}
\sound{-}
\licht{-}
\requisiten{-}

%Schlacht

\newpage
\section{Schlacht}
\label{sec:schlacht}
\charaktere{\Sum, \Gimli, \Gandalf, \Legolars, \Frodo, \Monk, \Beamtenzombies}
\setting{Dunkeler Gang}
\hauptbeamer{erst -, dann Youtube-Standbild}
\sound{-}
\licht{normal}
\requisiten{Trennwand, eventuell Pappkamaraden}
    
\regie{\Frodo, \Legolars und \Monk werden mit Licht und Soundeffekt hereingebeamt}

\begin{verseplay}[7em]
\s{\Frodo} Das Beamen kribbelt so schön.
\s{\Legolars} Das könnten die Katakomben unterhalb des Rektorats sein.
\end{verseplay}
\regie{\Gimli und \Sum kommen hinter einer Stellwand hervor gehüpft. Mit Partyhüten, Luftschlangen und Senktgläsern.}
\begin{verseplay}[7em]
\s{\Legolars} Wo wart ihr beiden?
\s{\Sum} Voll krasse Geschichte.
\s{\Gimli} Als wir durch das Star Gate gegangen sind waren wir zuerst in einer Vorlesung über - ich habe keinen blassen Schimmer, worum es ging. Auf jeden Fall waren wir danach - am besten schaut's euch auf Youtube an.
\end{verseplay}
\regie{Youtube-Standbild mit ``Dieses Video ist in Mittelerde nicht verfügbar''.}

\begin{verseplay}[7em]
\s{\Sum} So ein Mist aber auch.
\s{\Frodo} Auch egal, Hauptsache wir sind wieder vereint.
\end{verseplay}

\regie{Gandalf reitet plötzlich auf einem Steckenpferd herein, ``Look at my horse'' auf den Lippen.}

\begin{verseplay}[7em]
\s{\Frodo} Gaaaaaaaandaaaaaaalf. \kregie{\Frodo stürmt auf \Gandalf zu und knuddelt ihn leidenschaftlich.}
\s{\Legolars} Wir dachten du wärst TOT.
\s{\Gandalf} \kregie{zeigt Extraleben} Ich hatte noch ein Extraleben übrig.
\s{\Gimli} Ein Hoch auf \Elron und sein Schrottwichteln.
\s{\Gandalf} \kregie{\Gandalf holt Karte aus seinem Bart raus} Die Karte sagt, dass wir fast da sind, wir sollten uns beeilen!
\s{\Sum} Tolle Karte.
\s{\Gandalf} Da vorne links.\kregie{Weist mit Stab in die Richtung.}
\end{verseplay}

\regie{Sie schreiten voran. Nach ein Paar Schritten kommen aus dieser Richtung \Beamtenzombies.}

\begin{verseplay}[7em]
\s{\Gimli} \Beamtenzombies!
\s{\Gandalf} Geronimo. \kregie{\Gandalf rennt nach vorne.}
\end{verseplay}

\regie{Die Schlacht geht los. Monk monkt rum. Der Rest ignoriert Monk und kämpft. Dann wird ``Währenddessen auf der Brücke des ''Todessterns`` gezeigt. FIXME: Wie? Der Rest wird während der Probe entschieden. Zum Schluß gehen sie durch eine Tür mit der Aufschrift ''Druckerraum`` oder so.}

\regie{Licht aus.}

\newpage
\section{Zerstörung der Prüfungsordnung}
\label{sec:zerstoerung}
\charaktere{-}
\setting{-}
\hauptbeamer{-}
\sound{-}
\licht{-}
\requisiten{-}
    
\regie{FIXME: Whatever}

 %
% BEGIN: Abspann
%
\newpage
\section{Abspann}
  \label{sec:Abspann}
    \charaktere{}
     \setting{}
    \hauptbeamer{}
    \sound{}
    \licht{}
    \requisiten{}
    
    \regie{Video: Die Pr"ufungsordnung wird wieder aus dem M"ull geholt und zusammengeklebt. Text: Die Pr"ufungsordnung schl"agt zur"uck\\
    ABSPANN\\
    Die R"uckkehr der Inphima. Pr"ufungsordnung wird brutaler wieder kaputt gemacht.}
    
% END:
%
\newpage
\section{Ring}
\label{sec:ring}
\charaktere{alle Zestörer}
\setting{alles wie bei der Zerstörung}
\hauptbeamer{-}
\sound{erst -, dann Gangnam Style}
\licht{normal}
\requisiten{Gians aufgedrehter Swag}

\regie{Alles wie vor dem Abspann. FIXME: \Elron besitzt den Ring, den müssen wir noch \Legolars geben.}

\begin{verseplay}[7em]
\s{\Legolars} Wo ist eigentlich der Ring gut... \kregie{zieht Ring an}
\end{verseplay}

\regie{OPPAN GANGNAM STYLE!, alle tanzen Gangnam Style. Licht aus.}

\include{sketche/schluempfe}

\end{document}
