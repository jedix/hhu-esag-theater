\documentclass{scrartcl}
\usepackage[utf8]{inputenc}
\usepackage[T1]{fontenc}
\usepackage[ngerman]{babel}
\usepackage{play}
\usepackage{titlesec}
\usepackage{hyperref}
\usepackage{amsmath}
\usepackage{xspace}
\titleformat{\section}{\bf}{Szene \thesection:\quad}{0em}{}
\itemindent 0pt
%
%        BEGIN: Utilities
%
\newcommand{\s}[1]{\speaker #1}
\newcommand{\regie}{\longdirection}
\newcommand{\kregie}{\shortdirection}
\newcommand{\charaktere}[1]{\regie{\textbf{Charaktere:} #1}}
\newcommand{\setting}[1]{\regie{\textbf{B{\"u}hnenbild:} #1}}
\newcommand{\hauptbeamer}[1]{\regie{\textbf{Hauptbeamer:} #1}}
\newcommand{\sound}[1]{\regie{\textbf{Sound:} #1}}
\newcommand{\licht}[1]{\regie{\textbf{Licht:} #1}}
\newcommand{\requisiten}[1]{\regie{\textbf{Requisiten:} #1}}
\newcommand{\technik}{\xspace\shortdirection}
%g\newcommand{\technik}[1]{}
%
%        END: Utilities
%


%
%        BEGIN:  Charaktere
%

% Hauptcharaktere
\newcommand{\Frodo}{Frodo\xspace}
\newcommand{\Gandalf}{Gandalf\xspace}
\newcommand{\Sum}{Sum\xspace}
\newcommand{\Legolars}{Legolars\xspace}
\newcommand{\Gimli}{G1ml1\xspace}
\newcommand{\Monk}{Monk\xspace}
\newcommand{\Elron}{Elron\xspace}
\newcommand{\Galadriel}{Galadriel\xspace}
%Nebencharaktere
\newcommand{\Saruman}{Saruman\xspace}
\newcommand{\Pacman}{Pacman\xspace}
\newcommand{\Waldo}{Waldo\xspace}
\newcommand{\Firmenname}{Richtig tolle Firma GmbH und Co KG}
\newcommand{\Chor}{Chor\xspace}
\newcommand{\Alle}{Alle\xspace}

% QED
\newcommand{\QEDHost}{George \xspace}
\newcommand{\QEDGuestA}{QEDGuestA\xspace}
\newcommand{\QEDGuestB}{QEDGuestB\xspace}
\newcommand{\QEDGuestC}{QEDGuestC\xspace}

%
%        END:  Charaktere
%

\begin{document}
%
%        BEGIN: Titelseite
%
\title{ESAG-Theater 2012}
\author{}
\date{Letzte "Anderung: \today}
\maketitle
\tableofcontents
%
%        END:        Titelseite
%
%
%        BEGIN: Zusammenfassung Charaktere
%
\newpage
\textbf{Charaktere}
\begin{verseplay}[10em]
\s{\Frodo}
        \kregie{Schauspieler} \kregie{Szene \ref{sec:frodo-findet-po}}\\
        Kurzbeschreibung
\end{verseplay}

\include{sketche/band1}
    
\newpage
\section{Kazoo-Intro}
\label{sec:kazoo_intro}
\charaktere{Kazooisten}
\setting{-}
\hauptbeamer{THX-Logo, Dolby Analog}
\sound{-}
\licht{-}
\requisiten{-}
    
\regie{Die Kazooisten sind im Raum verteilt. Wenn das THX-Logo erscheint, spielen sie den THX-Sound. Danach spielen die das Harry-Potter-Intro.}


%
% BEGIN: SW HdR Intro
%
\newpage
\section{Intro}
  \label{Intro}
    \charaktere{Vorleser, }
     \setting{Rednerpult bei Band}
    \hauptbeamer{Video Intro abspielen}
    \sound{}
    \licht{}
    \requisiten{}
\regie{Text von Video:}
Es war einmalvor langer Zeit in einer weit, weit entfernten Galaxis... .\\
\regie{Star Wars Schriftzug}
Das Universium ist im Wandel\\
Ich sp"ure es im Vakuum \\
Ich sp"ure es im Raum\\
Ich rieche es im "Ather\\
\regie{nicht zu Nachahmungszwecken empfohlen}
Vieles was einst war ist verloren, da niemand mehr lebt, der sich erinnert.\\
Alles begann mit dem Schmieden der gro"sen Pr"ufungsordnungen...\\
\regie{Orignal HdR Video einspielen. Text auch ins Video., eventuell einzelne Bilder in Video.}
Drei wurden der Inphima gegeben, unsterblich, und die weisesten und reinsten aller Lebewesen.\\
Sieben den Medizinern, gro"se Quaksalber und geschickte Metzger, in ihren Hallen aus Backstein.\\
Und neun, neun Pr"ufungsordnungen wurden den Wirtschaftswissenschaftlern geschenkt,\\
die vor allem anderen nach Macht streben.\\
Denn diese Pr"ufungsordnungen bargen die St"arke und den Willen jede Fakult"at zu leiten.\\
Doch sie wurden alle betrogen!\\
Denn es wurde noch eine Pr"ufungsordnung gefertigt\\
In der Verwaltung, in der Tiefe des Rektorats, schmiedete der dunkle Rektor Sauron heimlich eine Rahmenpr"ufungsordnung,\\
um alle anderen zu beherrschen.\\
In diese Pr"ufungsordnung floss seine Grausamkeit, seine Bosheit, und sein Wille alle Freiheit zu unterdr"ucken.\\
\regie{Stimme aus dem Off als Chor:}
Eine Pr"ufungsordnung sie zu knechten,
sie alle zu finden, 
ins Dunkel zu treiben 
und ewig zu binden\\
 \regie{Einzelne Stimme aus dem Off:}
Min G$\hat{o}$r bauglo hain phain, \\
min G$\hat{o}$r chebo hain, \\
Min G$\hat{o}$r togo hain phain, \\
ned duir gwedho hain.\\
 Das war Elbisch.\\
 \regie{Sendung mit der Maus Melodie.}
%END: SW HdR Intro
%

%Frodo Findet Pr�fungsordnung

\newpage
\section{Frodo Findet Pruefungsordnung}
  \label{Frodo Findet Pruefungsordnung}
    \charaktere{\Frodo, \Gandalf }
     \setting{Poststelle}
    \hauptbeamer{nichts}
    \sound{}
    \licht{}
    \requisiten{Kaffee (Tasse), Briefkasten, Post, Pr�fungsordnung}
    
\regie{Frodo holt seine Hauspost ab und h�lt dabei einen Kaffe in der Hand. Er findet in seinem Postfach die Rahmenpr�fungsordnung.}
\s{\Frodo} 




%ideen:
%Steht am Briefkasten und sortiert seine Post. Rechnungen, Mahnungen, Spam, Spam, Spam, Bacon, Eggs and Spam.
%Ein gro�er Umschlag, hmmm, ans Dekanat, muss wichtig sein, schulterzuck, auf.
%Gandalf kommt an. Wat is dat denn? bla bla bla Rahmenpr�fungsordnung bla bla vernichtet werden.
%Evtl. Bier und Erdn�sse im Bart (Anhalteranspielung). Muss vernichtet werden bla. Da wo sie erstellt wurde, im Rektorat.
%K�nnte sehr gef�hrlich sein, ich erkundige mich bei meinem Dekan, du ruft schon mal ne SP-Saal-Sitzung (sic!) ein.
\newpage
\section{Treffen zwischen Gandalf und Saruman}
\label{sec:gandalf-saruman}
\charaktere{\Gandalf, \Saruman, \Tuermonteura, \Tuermonteurb}
\setting{Sarumans Büro}
\hauptbeamer{-}
\sound{erst -, später Lichtschwerteffekt mit anschließendem ``Duel of the fates''}
\licht{normal, später UV}
\requisiten{Wanderstöcke/Lichtschwerter (UV-reaktiv lackierte Besenstile)}
\regie{\Saruman sitzt am Tisch, mehrere Papierstapel stehen vor ihm. Laptop steht vor ihm, tippt, wirkt beschäftigt. Plasmalampe steht daneben.}
\regie{Unsichtbare Tür wird auf die Bühne getragen und montiert. Montage wird überprüft. Monteure gehen ab. \Gandalf betritt die Bühne, klopft an unsichtbare Tür.}
\begin{verseplay}[7em]
\s{\Saruman} Wer ist da?
\s{\Gandalf} Feuer
\s{\Saruman} Feuer-wer?
\s{\Gandalf} \kregie{öffnet Tür einen Spalt, steckt Kopf durch} Wo brennts denn?
\s{\Saruman} Aaaaarrrrrgggghhhhh... \kregie{Facepalm}
\end{verseplay}
\regie{Gandalf öffnet die Tür ganz und tritt ein.}
\begin{verseplay}[7em]
\s{\Saruman} Je-des-mal. Was kann ich für dich tun, Ghandi?
\s{\Gandalf} Es gibt noch eine weitere Prüfungsordnung, der Rektor hat uns alle betrogen. Sie dient der Unterdrückung allen Lebens, alle anderen Prüfungsordnungen sind ihr unterstellt.
\s{\Saruman} Ich weiß.
\s{\Gandalf} \kregie{verwundert} Du weißt?
\s{\Saruman} Du siehst das alles falsch, sie kann uns nutzen. Wir können sie für unsere Zwecke einsetzen und alle Wirtschaftswissenschaftler, Juristen und Mediziner verbannen.
\s{\Gandalf} Der Rektor wird niemals etwas von seiner Macht an die Studierenden abtreten, auch nicht an den AstA-Vorsitzenden.
\s{\Saruman} Es ist zu spät, ich habe sie schon in die Hauspost gegeben.
\s{\Gandalf} Ach so ist das... ehh... Ich geh dann mal.
\s{\Saruman} Haaaaaaalt Stop! Schließe dich der dunklen Seite der Macht an, zusammen können wir die Universität beherrschen.
\s{\Gandalf} Ich werde mich dir niemals anschließen.
\s{\Saruman} Dann musst du sterben.
\end{verseplay}
\regie{Lichtschwerter werden gezückt. UV-Licht an. Soundeffekt. Musik: ``Duel of the fates''. Epischer Lichtschwertkampf. Und mit episch meine ich episch. Echt jetzt. Mit schwarz gekleideten Leuten, die bei Salti helfen und so. \Saruman stirbt. \Gandalf geht ab.}

\regie{Licht aus.}

\newpage
\section{SP-Treffen}
\label{sec:sp-treffen}
\charaktere{\Frodo, \Sum, \Gimli, \Legolars, \Elron, \Gandalf, \Galadriel}
\setting{Riesentischkreis}
\hauptbeamer{-}
\sound{-}
\licht{normal oder Spot auf Tisch}
\requisiten{Tisch, Stühle, PO-Kette, Laptops, Schrottwichtelgeschenke}
    
\regie{Das Studierendenparlament tagt. TOP-Liste:\\
TOP 0: Regularia\\
TOP 1: AstA-Berichte\\
TOP n: Schrottwichteln\\
TOP n+1: Sonstiges
}

\begin{verseplay}[7em]
\s{\Elron} Herzlich Willkommen zur 34. Sondersitzung des Studierendenparlamentes, stattfindend zwischen der regulären 1465. und 1466. Sitzung, auf Antrag von \Frodo Beutlin, Student der mathematisch-naturwissenschaftlichen Fakultät, fernmündlich eingereicht über die Facebook-Pinnwand des Studierendenparlamentes. Ich schlage \Galadriel als Protokollführende vor. Gibt es Gegenstimmen? \kregie{Pause} Gut. Kommen wir zur Feststellung der Anwesenheit. \Galadriel ist anwesend. \Sum?
\end{verseplay}
\regie{\Galadriel tippt die ganze Zeit über Protokoll.}
\begin{verseplay}[7em]
\s{\Sum} Ja.
\s{\Elron} G.. 1.. m.. l.. 1..?
\s{\Gimli} Das wird Gimli ausgesprochen, jo.
\s{\Elron} \Legolars?
\s{\Legolars} Anwesend. \kregie{meldet sich mit Bogen}
\s{\Elron} \Gandalf?
\s{\Frodo} Der kommt nach.
\s{\Elron} Wie immer. \Frodo?
\s{\Frodo} Nö.
\s{\Elron} Echt jetzt? \Saruman? \kregie{Pause} Der AstA-Vorstand ist wie üblich nicht anwesend. Ich stelle fest, dass wir trotzdem beschlussfähig sind. Gibt es Wünsche zur Änderung der bekanntgegebenen TOP-Liste? \kregie{Pause, \Frodo guckt ausdruckslos in der Gegend herum} \Frodo?
\s{\Frodo} Äh, ja. \Gandalf hat gesagt, ich solle schonmal eine Sondersitzung einberufen, es geht um irgendeine neue Rahmenprüfungsordnung.
\s{\Elron} Eine neue Prüfungsordnung wäre doch in den amtlichen Bekanntmachungen aufgeführt worden?
\s{\Frodo} \kregie{zuckt mit den Schultern}
\s{\Elron} Nun gut, da Gandalf noch nicht erschienen ist, füge ich den TOP Rahmenprüfungsordnung ans Ende der TOP-Liste vor Sonstiges an. Weitere Änderungswünsche? \kregie{Pause} Gut. TOP 1, Berichte des AstA. \kregie{Pause} Offensichtlich hat der AstA heute nichts zu berichten.
\end{verseplay}
\regie{Gandalf stürmt auf Steckenpferd herein.}
\begin{verseplay}[7em]
\s{\Gandalf} Wir sind alle dem Untergang geweiht! Der Rektor hat \kregie{wird HIER unterbrochen} eine neue Rahmenprüfungsordnung...
\s{\Elron} Einen Moment. Ich stelle hiermit für das Protokoll \Gandalf{}s verspätete Anwesenheit fest. Wärest du pünklich erschienen, so wäre dir bekannt, dass der TOP Rahmenprüfungsordnung an das Ende der TOP-Liste gesetzt wurde.
\s{\Gandalf} Aber aber aber...
\s{\Elron} \kregie{unbeeindruckt} Wir fahren nun fort mit TOP n: Schrottwichteln.
\end{verseplay}
\regie{\Gandalf setzt sich grummelnd hin.}
\begin{verseplay}[7em]
\s{\Frodo} Ich habe doch garnichts fürs Schrottwichteln dabei.
\s{\Elron} Das Schrottwichteln ist eine lange ungebrochene Tradition des Studierendenparlamentes. \kregie{funkelt \Frodo böse an}
\s{\Sum} Guck einfach, ob du irgendwas in deinen Taschen dabei hast.
\end{verseplay}
\regie{\Frodo kramt. Jeder liegt verdeckt etwas in eine Kiste, die auf den Tisch gestellt wird. In der Kiste: Goldkette, blaues Schwert, Limburger-und-Sardellen-Sandwich (von \Frodo), 1-UP-Pilz, goldener Ring mit komischer elbischer Inschrift und Benzin für die Kettensäge.}
\regie{\Gandalf steht auf, zieht 1-UP-Pilz aus der Kiste.}
\begin{verseplay}[7em]
\s{\Gandalf} Uhhhh, ein Pilz. \kregie{setzt sich wieder}
\end{verseplay}
\regie{\Sum steht auf, zieht blaues Schwert aus der Kiste.}
\begin{verseplay}[7em]
\s{\Gimli} Was ist das?
\s{\Sum} Ein blaues Schwert.
\s{\Gimli} Und was tut es?
\s{\Sum} Es leuchtet blau. \kregie{setzt sich wieder hin}
\end{verseplay}
\regie{\Frodo steht auf, zieht Goldkette aus der Kiste.}
\begin{verseplay}[7em]
\s{\Frodo} Oh, eine Goldkette. \kregie{setzt sich wieder hin}
\end{verseplay}
\regie{\Gimli steht auf, zieht Limburger-und-Sardellen-Sandwich.}
\begin{verseplay}[7em]
\s{\Gimli} \kregie{empört} Was ist das denn?
\s{\Frodo} Ehmmm... das erkenne ich sofort: Ein Limburger-und-Sardellen-Sandwich. Ein Bissen, und du musst 4 Tage lang nichts essen.
\s{\Elron} \kregie{wirft ein} \emph{Möchtest} 4 Tage lang nichts essen.
\end{verseplay}
\regie{\Gimli setzt sich wieder.}
\regie{\Legolars steht auf, zieht Benzin.}
\begin{verseplay}[7em]
\s{\Legolars} \kregie{Liest vor} Benzin für die Kettensäge... \kregie{enttäuscht} Toll, Nahkampfwaffenmunition. \kregie{setzt sich wieder hin}
\end{verseplay}
\regie{\Elron steht auf, zieht Ring.}
\begin{verseplay}[7em]
\s{\Elron} Oh, ein goldener Ring mit elbischer Inschrift.
\s{\Sum} Was steht drauf? Was steht drauf?
\s{\Elron} Es ist in der dunklen Sprache geschrieben. \kregie{kurze Pause} Made in Bangladesh.
\end{verseplay}
\regie{\Elron steckt den Ring in die Tasche, räumt die Kiste vom Tisch, setzt sich wieder.}
\begin{verseplay}[7em]
\s{\Elron} Nun, nachdem die Tradition des Schrottwichtelns bei Sitzungen, auch außerordentlichen, des Studierendenparlamentes gewahrt wurde, kommen wir zum nächsten TOP: Rahmenprüfungsordnung und Zerstörung der Welt.
\s{\Gandalf} Wie \Frodo euch sicher bereits mitgeteilt hat, hat der Rektor ohne das Wissen der Studenten eine weitere Prüfungsordnung erschaffen.
\s{\Frodo} zeigt die Prüfungsordnung
\s{\Gandalf} Diese wird, wenn sie in Kraft tritt, über allen anderen Prüfungsordnungen stehen und mit ihren Regeln alles Leben unterwerfen. Das Studentenleben, wie wir es kennen, wird dann nicht mehr möglich sein.
\s{\Elron} Nein.
\s{\Gandalf} Doch.
\s{{Chor}} Ohhhh!
\s{\Elron} \kregie{arrogant} Ohne die Unterschrift des AstA-Vorsitzenden wird diese Prüfungsordnung nicht in Kraft treten können.
\s{\Gandalf} Der AstA-Vorsitzende ist uns keine Hilfe. Er hat bereits unterzeichnet.
\s{\Elron} Nein.
\s{\Gandalf} Doch.
\s{{Chor}} Ohhhh!
\s{\Frodo} Kann ich sie nicht einfach kaputt machen? In den Müll werfen?
\s{\Gandalf} Nein. Sie kann nur da zerstört werden, wo sie geschaffen wurde. Wir müssen ins Rektorat.
\s{\Elron} Man geht nicht einfach ins Rektorat.
\s{\Gimli} Dann brauchen wir einen Plan. Und eine Gruppe wackerer Helden.
\s{\Elron} Ich werde von hier aus mit \Galadriel den Widerstand organisieren.
\s{\Frodo} Ich habe die Prüfungsordnung, ich werde sie auch vernichten. \kregie{bindet Prüfungsordnung an Kette}
\s{\Sum} YOLO! Ich bin dabei.
\s{\Gandalf} Auf mich werdet ihr nicht verzichten können, ich kenne alle Wege zum Rektorat.
\s{\Legolars} Du kannst mit meinem Bogen rechnen. \kregie{Hält Bogen hoch}
\s{\Gimli} Und mit meiner Axt! Und meinem Sandwich! \kregie{Hält erst Axt, dann auch Sandwich hoch}
\end{verseplay}

\regie{Licht aus.}

%Nach Moria

\newpage
\section{Nach Moria}
\label{sec:nach-moria}
    \charaktere{\Frodo, \Sum, \Gimli, \Legolars, \Gandalf}
    \setting{Mario-Hintergründe, bewegt}
    \hauptbeamer{nichts}
    \sound{nichts, nachher Supersterneffekt}
    \licht{Spot}
    \requisiten{Mario-Hintergründe}
    
\begin{verseplay}[10em]

\regie{Sie laufen durch den Wald. Sie laufen durch den Wald. Sie laufen durch den Wald. Sie laufen durch den Wald. Sie laufen durch den Wald.}

\s{\Sum} Sind wir schon da?
\s{\Gandalf} Nein.

\regie{Werbepause, z.B. Mensa-Card, danach Mate, evtl. mehr}

\s{\Sum} Sind wir jetzt da?
\s{\Gandalf} Nein.

\regie{Jetzt nur ein Spot, z.B. Klingeltöne}

\s{\Sum} Sind wir jetzt da?
\s{\Gandalf} Ja.
\s{\Sum} \kregie{Aufgeregt} Wirklich?
\s{Alle} NEIN!

\s{\Legolars} Wo gehen wir eigentlich lang?
\s{\Gandalf} Über den großen Mensabrückenpass.
\s{\Gimli} Wir können auch bei meinem Vetter Moria vom Nerdpol durch den LAN-Partykeller gehen. Ein altes System aus Gängen führt von dort aus direkt zur Verwaltung.
\s{\Sum} Gibts da auch was zu essen?
\s{\Gimli} Dort wird uns ein Fest geboten, mit Tunefisch aus der Nerdsee.
\s{\Sum} Mhhhhhhmmm, lecker.
\s{\Gandalf} Dann auf zu Moria.

\regie{Sie laufen durch den Wald. Gumbas und Kröten laufen vorbei. Sie springen drüber.}

\s{\Sum} Das dauert aber lange.

\regie{Eine Fragezeichenbox erscheint. Gandalf schlägt seinen Laserstock dagegen. Ein Stern springt heraus. Frodo fängt ihn. Mario-Stern-Melodie, alle gehen sehr schnell. Nach einigen Sekunden abblenden, gleichzeitig Kartenvideo auf Beamer, dabei wird ein Tor aufgebaut. Blende zurück. Sie stehen erschöpft vor dem Tor.}

\s{\Gimli} Wir haben jetzt kurz nach drei. Um diese Zeit schläft mein Vetter eh noch. Lasst uns erst einmal rasten.
\s{\Sam} Aber ich will zum Festmahl!
\s{\Legolars} Gandalf, kannst du den Einbruch der Dunkelheit nicht beschleunigen?
\regie{Gandalf stampf seinem Laserstock drei mal auf. Es wird langsam dunkel. UV-Licht geht langsam an. Captcha in elbischer Schrift auf der Tür erscheint.}

\s{\Frodo} Was ist denn da erschienen?
\s{\Gimli} Das ist das Einbruchsschutzcaptcha meines Vetters. Er schützt sich davor vor Wirtschaftswissenschaftlern.
\s{\Gandalf} \kregie{liest vor} Wer das liest ist doof.
\s{\Legolars} \kregie{tippt ein} Funktioniert nicht.
\s{\Gandalf} Klein und zusammen geschrieben.

\regie{\Legolars tippt ein. UV aus. Tür öffnet sich.}

\end{verseplay}

\newpage
\section{Moria}
\label{sec:moria}
\charaktere{\Frodo, \Sum, \Gimli, \Legolars, \Gandalf, \Monk, \Leichen}
\setting{Dunkle Höhle, Tische mit PCs und leblose Körper}
\hauptbeamer{-}
\sound{-}
\licht{erst Spot, dann normal}
\requisiten{Lichtschalter, Tische, PCs}
    
\regie{Monk sitzt die ganze Zeit in der Ecke und wippt ``Tuch, Tuch''. Die Gruppe kommt auf die Bühne. Spot auf die Gruppe, der Rest ist abgedunkelt.}

\begin{verseplay}[7em]
\s{\Sum} Mann, ist das hier dunkel. Man sieht seine eigenen Füße nicht.
\s{\Gandalf} Lumos! \kregie{schaltet seinen Stab ein und wedelt pottermäßig rum}
\s{\Gimli} Das geht auch einfacher. \kregie{schaltet großen Lichtschalter ein}
\end{verseplay}

\regie{Licht an. Alle erschaudern, untersuchen die leblosen Körper.}

\begin{verseplay}[7em]
\s{\Gandalf} \kregie{schaut von einer Leiche hoch} Das ist keine LAN-Party, das ist ein Grab!
\s{\Gimli} \kregie{fällt auf die Knie} Neeeeeiiiiiiin!
\s{\Frodo} Was ist hier passiert?
\s{\Sum} Sie sind tot, sie sind alle tot.
\end{verseplay}

\regie{\Sum stupst an den Leichen rum. Während des nächsten Gespräches mit \Monk setzt sich \Sum unauffällig an einen Rechner. \Frodo guckt ebenso unauffällig zu.}

\begin{verseplay}[7em]
\s{\Legolars} \kregie{entdeckt Monk, zeigt hin} Guckt mal, da hinten lebt noch einer!
\s{\Gandalf} \kregie{geht hin, klopft Monk auf die Schulter} Hallo?
\s{\Monk} \kregie{wischt hektisch seine Schulter} Nicht berühren! \kregie{hüpft weg}
\s{\Gandalf} Wer bist du? Was ist hier passiert?
\s{\Monk} Haben sie ein Tuch? Ich brauche ein Tuch.
\s{\Gandalf} \kregie{guckt sich suchend um}
\s{\Legolars} \kregie{kommt, gibt Tuch}
\s{\Monk} Vielen Dank! \kregie{säubert sich}
\s{\Monk} Mein Name ist Adrian Monk. Könnte ich noch ein Tuch bekommen?
\s{\Legolars} \kregie{gibt Tuch}
\s{\Monk} \kregie{säubert sich weiter} Danke. So wie ich das rekonstruieren kann, hat hier vor viereinhalb Wochen eine LAN-Party stattgefunden. Erst haben sie noch jeden Tag Pizza bestellt, aber dann waren sie so sehr in ihr Spiel vertieft, dass sie vergessen haben, zu essen.
\s{\Gimli} \kregie{seufzt und schluchzt stärker}
\s{\Legolars} Und was machen sie hier?
\s{\Monk} Ich war vorletzte Woche mit meiner Assistentin Natalie auf dem Weg zur Universitäts- und Landesbibliothek. Wir wollten eigentlich über den großen Mensabrückenpass gehen, aber dann wären wir nicht immer schön abwechselnd links und rechts abgebogen. Deshalb sind wir durch diesen Keller gegangen und haben die Leute hier so vorgefunden.
\s{\Gandalf} Das war vor zwei Wochen, was machst du immernoch hier und wo ist deine Assistentin?
\s{\Monk} Haben sie gesehen, wie das hier aussieht? Ich musste selbstverständlich erstmal hier aufräumen. Überall Pizzakartons, Tetra-Packs, nicht nach Geschmack sortiert, die Leichen lagen einfach verstreut in der Gegend herum, überall Staub und Fettflecken und man kann hier noch nicht mal lüften. Letzte Woche war es Natalie zu viel und sie ist aus unerfindlichen Gründen gegangen. Aber das Schlimmste war, dass mir vor drei Tagen die Taschentücher und Sagrotan ausgegangen sind. \kregie{zu \Legolars} Haben sie noch ein Tuch für mich?
\s{\Legolars} \kregie{reicht ihm ein Tuch}
\s{\Monk} \kregie{putzt weiter}
\s{\Sum} Hey, das Spiel ist echt gut!
\s{\Gandalf} \kregie{geht zu zerstörtem \Gimli} Wir müssen weiter, bevor einige von uns \kregie{zeigt mit dem Auge auf \Sum} auch dem Spiel verfallen.
\s{\Gimli} \kregie{nickt bedrückt und richtet sich auf}
\s{\Gandalf} \kregie{bestimmt} Wir gehen weiter!
\s{\Sum} \kregie{wehrt sich, wird von \Frodo mitgezogen}
\end{verseplay}

\regie{Alle ausser \Monk gehen langsam nach rechts ab, \Monk zögert, folgt monkig mit den Worten ``Tuch, Tuch, Tuch''. Licht aus.}

%Moria 2

\newpage
\section{Moria 2}
\label{sec:moria2}
    \charaktere{\Frodo, \Sum, \Gimli, \Legolars, \Gandalf, \Monk, \Pacman}
    \setting{Dunkle Höhle, Stellwände auf beiden Seiten}
    \hauptbeamer{nichts}
    \sound{nichts}
    \licht{}
  \requisiten{2 Stellwände}
    
\regie{Die Party betritt die Bühne von hinter Stellwand(a). Monk monkt rum.}

\begin{verseplay}[10em]
\s{\Gandalf} Ich spüre eine Erschütterung der Macht.
\s{\Sum} Vielleicht hast du einfach nur Hunger.
\s{\Gimli} \kregie{sarkastisch} Möchtest du mein Sandwich?
\s{\Frodo} Das ist wirklich lecker! Und fast frisch!
\s{\Legolars} \kregie{Die Idioten ignorierend} Ich spüre auch etwas. Gandalf?
\s{\Gandalf} \kregie{Kunstpause} Es ist ein Dämon aus der alten Zeit. \kregie{kurze Pause} Lauft!
\end{verseplay}

\regie{Pacman erscheint waka-waka-end aus der Stellwand(a) und jagt die Party hinter die Stellwand(b). Licht aus.}
%Moria 3

\newpage
\section{Moria 3}
\label{sec:moria3}
    \charaktere{\Frodo, \Sum, \Gimli, \Legolars, \Gandalf, \Monk, \Pacman}
    \setting{Dunkle Höhle, Stellwände auf beiden Seiten}
    \hauptbeamer{nichts}
    \sound{nichts}
    \licht{}
  \requisiten{2 Stellwände}
    
\regie{Die Party, Gandalf als letzter, hetzt auf die Bühne von hinter der Stellwand(a). Monk monkt rum. Gandalf hält auf Mitte der Bühne an, die anderen halten verzögert. Gandalf dreht sich zu Pacman. Pacman stoppt. Während Gandalfs Monolog weicht Pacman langsam zurück.}

\begin{verseplay}[10em]
\s{\Gandalf} Ich bin ein Diener der Wissenschaften und Gebieter über die Gesetze der Natur. Du - kannst - nicht - vorbei! \kregie{schlägt mit Stock auf den Boden}
\s{\Pacman} \kregie{Weicht kurz stärker zurück, taumelt. Sprintet waka-waka-end los.}
\s{\Gandalf} Flieht ihr Narren!
\end{verseplay}

\regie{Die Party flieht hinter Stellwand(b). Pacman frisst Gandalf, indem er vor ihn rennt und Gandalf sich klein macht. Licht aus.}

\begin{verseplay}[10em}
\s{\Frodo} Gandaaaaalf!
\end{verseplay}

\newpage
\section{Moria 4}
\label{sec:moria4}
\charaktere{\Frodo, \Sum, \Gimli, \Legolars, \Monk}
\setting{Dunkle Höhle, Stellwand, Star Gate}
\hauptbeamer{nichts}
\sound{nichts}
\licht{normal}
\requisiten{Stellwand(a), Star Gate}
    
\regie{Die Party läuft von hinter Stellwand(a) und stoppt in der Mitte. \Monk monkt rum.}

\begin{verseplay}[7em]
\s{\Legolars} Ich glaube wir sind hier erstmal in Sicherheit.
\end{verseplay}

\regie{\Frodo und \Sum hocken sich hin und tuscheln bedröppelt.}

\begin{verseplay}[7em]
\s{\Gimli} \kregie{guckt sich um, entdeckt das Star Gate} Guck mal \Legolars, ich glaube hier gehts raus!
\s{\Legolars} \kregie{geht hin, untersucht das Star Gate, kurze Pause, leicht überrascht} Das ist ein Star Gate.
\s{\Gimli} Kannst du es bedienen?
\s{\Legolars} \kregie{dreht das Glücksrad} Sieht gut aus. \Gimli, \Sum, geht ihr als erstes.
\end{verseplay}

\regie{\Sum und \Frodo stehen auf. \Gimli und \Sum gehen durch das Star Gate. \Monk monkt am Gerät rum.}

\begin{verseplay}[7em]
\s{\Monk} Das ist ja total verstaubt. \kregie{entstaubt Glücksrad, Glücksrad verstellt sich}
\s{\Legolars} Neiiiiin, was hast du getan?
\s{\Monk} Es war dreckig.
\s{\Legolars} Du hast die Adresse verstellt! Egal, wir müssen hier weg. Kommt!
\end{verseplay}

\regie{\Frodo, \Monk und \Legolars gehen durch das Star Gate. Licht aus.}


 %
% BEGIN: Abspann
%
\newpage
\section{Abspann}
  \label{sec:Abspann}
    \charaktere{}
     \setting{}
    \hauptbeamer{}
    \sound{}
    \licht{}
    \requisiten{}
    
    \regie{Video: Die Pr"ufungsordnung wird wieder aus dem M"ull geholt und zusammengeklebt. Text: Die Pr"ufungsordnung schl"agt zur"uck\\
    ABSPANN\\
    Die R"uckkehr der Inphima. Pr"ufungsordnung wird brutaler wieder kaputt gemacht.}
    
% END:
%

%
%        BEGIN:  Band, Ende
%
\newpage
\section{Band, Ende -- Schlumpflied}
\label{sec:BandTeilA}
        \charaktere{}
        \setting{}
        \hauptbeamer{}
        \sound{}
        \licht{}
        \requisiten{}
\regie{8 Leute mit Nummern auf dem R"ucken f"ur Modulo Erkl"arung}
%        END: Band, Ende
%

\include{sketche/qed}

\end{document}
